Toutes les formules de trigonométrie classiques peuvent se démontrer en utilisant les propriétés de l'exponentielle complexe. Tous les symboles non définis désignent des réels. À faire en exercice.

\paragraph{Cercle}
\begin{remarque}
On rappelle que l'équation du cercle, ce n'est rien d'autre qu'une reformulation du théorème de Pythagore
\end{remarque}
Un point de coordonnées $(\cos x,\sin x)$ appartient au cercle unité :

\[ \cos^2x+\sin^2x = 1\]
\begin{remarque}
Les expressions $\sqrt{1-\cos^2 x}$ et $\sqrt{1 - \sin^2 x}$ doivent être reconnues instantanément. Attention aux pièges, voir exercice ci-dessous.
\end{remarque}
\begin{exo}
Montrer que l'assertion
\[ \forall x\in \R, \sqrt{1-\cos^2 x} = \sin x\]
est... fausse. Quelle est la formule correcte ?
\end{exo}


\paragraph{Somme}

\begin{align*}
\cos(a+b) &= \cos a \cos b - \sin a \sin b\\
\sin(a+b )&= \sin a \cos b + \cos a \sin b\\
\end{align*}

On en déduit, en appliquant ces formules lorsque $b=a$, les formules pour l'angle double:

\begin{align*}
\cos(2a) &= \cos^2 a - \sin^2 a \\
&= 2\cos^2 a-1 \\
&= 1-2\sin^2 a\\
\sin(2a)&= 2\sin a \cos a \\
\end{align*}

\begin{exo}
Trouver une formule pour $\cos(3a)$ et $\sin(3a)$.
\end{exo}

\paragraph{Symétries de translation}

Les fonctions cosinus et sinus sont $2\pi$-périodiques.
\begin{remarque}
Géométriquement, ceci signifie que leur graphe est invariant par translation de $2\pi$ selon l'axe des abscisses.
\end{remarque}


\begin{align*}
\cos(x+2\pi)&=\cos x\\
\sin(x+2\pi)&=\sin x
\end{align*}

\paragraph{Symétries axiales}

La fonction cosinus est paire: $ \cos(-x)=\cos(x)$.
\begin{remarque}
Une fonction est paire ssi son graphe admet la droite d'équation $x=0$ comme axe de symétrie.
\end{remarque}
\begin{exo}
Montrer que la fonction sinus n'est pas paire.
\end{exo}

Le graphe du sinus admet la droite (verticale) d'équation $x=\pi/2$ comme axe de symétrie :
\[ \sin(\pi-x)=\sin(x) \]

\paragraph{Symétries centrales}

La fonction sinus est impaire : $\sin(-x)=-\sin(x)$.
\begin{remarque}
Une fonction est impaire ssi son graphe admet le point de coordonnées $(0,0)$ comme centre de symétrie.
\end{remarque}
\begin{exo}
Montrer que la fonction cosinus n'est pas impaire.
\end{exo}

Le point de coordonnées $(\pi/2,0)$ est un centre de symétrie pour le graphe du cosinus, ce qui s'écrit de la façon suivante:
\[ \cos(\pi-x)=-\cos(x)\]


\begin{exo}
Déduire des formules de parité et imparité une formule pour $\cos(a-b)$ et $\sin(a-b)$. (À apprendre par c\oe ur également.)
\end{exo}

\paragraph{Symétries glissées\footnote{Une symétrie glissée est une symétrie axiale suivie d'une certaine translation}}
\begin{exo}
Montrer que les fonctions sinus et cosinus ne sont pas $\pi$-périodiques.
\end{exo}

\begin{align*}
\cos(x+\pi) &= -\cos x\\
\sin(x+\pi) &= -\sin x\\
\end{align*}

Ces formules se déduisent des précédentes.

\paragraph{Passage du cosinus au sinus}

\[ \cos(x)=\sin(\pi/2-x)\]
\[ \sin(x)=\cos(\pi/2-x)\]



