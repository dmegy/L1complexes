\begin{definition}
En coordonnées dans une base qui sera alors orthonormée par définition
\end{definition}

\begin{proposition}
Changement de repère orthonormé.
\end{proposition}

\begin{proposition}
Écriture en coordonnée complexe.
\end{proposition}

\begin{proposition}
Bilinéarité, symétrie, défini positif.
\end{proposition}

\begin{definition}
Norme d'un vecteur.
\end{definition}

On peut donc définir la norme d'un vecteur grâce au produit scalaire. Mais l'inverse est également vrai  : on peut entièrement caractériser le produit scalaire uniquement grâce aux normes de vecteurs : 

\begin{proposition}
[Identités de polarisation]

\end{proposition}

\begin{definition}
Orthogonalité de deux vecteurs.
\end{definition}

\begin{proposition}
Cauchy-Schwarz
\end{proposition}

\begin{proposition}
Inégalité triangulaire
\end{proposition}

