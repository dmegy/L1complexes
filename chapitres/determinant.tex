\begin{definition}
Déterminant de deux vecteurs, relativement à une base.
\end{definition}

\begin{proposition}
[Écriture en coordonnée complexe]
Soient $\vec u$ et $\vec v$ deux vecteurs du plan, d'affixes $z$ et $z'$ relativement à une base $\mathcal B$. Alors, leur déterminant (relativement à cette base) est
\[ \det(\vec u, \vec v) = \Im(\overline z\cdot z')\]
\end{proposition}
\begin{proof}
Soient $(x,y)=\Coord \vec u$ et $(x',y')=\Coord \vec v$. Alors
\[ \Im(\overline z\cdot z') = \Im\left(xx'+yy'+i(xy'-yx')\right) = xy'-yx' = \det(\vec u, \vec v).\]
\end{proof}

\begin{definition}
On dit que deux bases ont la même orientation si le déterminant de l'une dans l'autre est positif. SInon on dit qu'elles ont une orientation inverse.
\end{definition}

\begin{exemples}Soit $\mathcal B = (u,v)$ la base canonique de $\R^2$.
\begin{enumerate}
\item La base $\mathcal B'=(\vec u+\vec v, -\vec u+2\vec v)$ a la même orientation que $\mathcal B$.
\item La base $\mathcal B''=(\vec v, \vec u)$ a l'orientation inverse de celle de $\mathcal B$.
\end{enumerate}
\end{exemples}

Orientation, bases directes, indirectes

\begin{proposition}
La formule pour l'écriture du déterminant est la même dans toute base orthonormée directe.
\end{proposition}

\begin{proposition}
\begin{enumerate}
\item (bilinéarité)
\item (Le déterminant est une forme bilinéaire alternée)
\item (Le déterminant est une forme bilinéaire antisymétrique)
\end{enumerate}
\end{proposition}

Interprétation géométrique : condition de colinéarité.

Interprétation géométrique : $\lvert \det(\vec u, \vec v)\rvert $ est l'aire du parallélogramme porté par $\vec u$ et $\vec v$. (Sans les valeurs absolues, on obtient l'aire algébrique, qui tient compte de l'orientation, et vérifie de meilleures formules.)

\begin{exemple}
Soit $ABC$ le triangle dont les sommets ont pour affixe $a=1+i$, $b=5+2i$ et $c=3+4i$. Alors il a une aire égale à 
\[\mathcal A(ABC)= \frac12\left|\Im \overline{(b-a)}(c-a)\right| = \frac12\left|\Im (4-i)(2+3i)\right|=5\]
\end{exemple}

