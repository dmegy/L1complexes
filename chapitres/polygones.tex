\subsection{Triangles}

\subsection{Quadrilatères}

Dans ce qui suit, on suppose que tous les quadrilatères ont des sommets distincts.

Un quadrilatère peut être \emph{croisé} ou \emph{non-croisé}.

\begin{definition}
Un quadrilatère est dit \emph{convexe} si, lorsque $M$ et $P$ sont des points intérieurs au quadrilatère, le segment $[MP]$ est intérieur au quadrilatère.
\end{definition}

\begin{propdef}
Soit $ABCD$ un quadrilatère non croisé. Les assertions suivantes sont équivalentes:
\begin{enumerate}
\item Les côtés opposés sont parallèles deux à deux.
\item $AB=CD$ et $BC=AD$.
\item $AB=CD$ et $(AB)//(CD)$.
\item $\vec{AB}=\vec{DC}$.
\item $BC=AD$ et $(BC)//(AD)$.
\item $\vec{BC}=\vec{AD}$.
\item Les diagonales se coupent en leur milieu.
\end{enumerate}

Lorsque ces conditions sont vérifiées, on dit que $ABCD$ est un \emph{parallélogramme}.
\end{propdef}

\begin{definition}
Un quadrilatère non croisé $ABCD$ est un \emph{trapèze} s'il possède deux côtés parallèles.
\end{definition}

\begin{definition}
Un quadrilatère non croisé $ABCD$ est un rectangle si ses quatre angles sont droits.
\end{definition}

\begin{definition}
Un quadrilatère non croisé est un \emph{losange} si ses quatre côtés ont la même longueur.
\end{definition}

\begin{definition}
Un quadrilatère qui est à la fois un rectangle et un losange est un \emph{carré}.
\end{definition}