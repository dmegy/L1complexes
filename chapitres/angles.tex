\subsection{Angles géométriques (entre deux demi-droites)}

Cauchy-Schwarz permet de définir l'angle géométrique de deux vecteurs non nuls $\vec u$ et $\vec v$ comme

le réel $\arccos\left(\frac{\vec u\cdot \vec v}{\|\vec u\|\cdot \|\vec v\|}\right)$, puisque la parenthèse est d'après CS un réel entre $-1$ et $1$, donc dans le domaine de définition d'arccos.

Cette notion donne un objet de nature simple (un réel), qui plus est facile à calculer : 

\begin{exemples}
\end{exemples}

Par contre, la notion de comporte mal, elle ne vérifie pas de formules pratiques, elle ne permet pas de démontrer des théorèmes facilement...



\subsection{Angles orientés}

Notation
Cahier des charges : 
\begin{enumerate}
\item remplacement par vecteurs unitaires.
\item Chasles.
\item $\widehat{(\overrightarrow u, \overrightarrow u)} \equiv 0 \mod 2\pi$.
\item $\widehat{(\overrightarrow u, \overrightarrow v)} \equiv -\widehat{(\overrightarrow v, \overrightarrow u)} \mod 2\pi$.
\end{enumerate}

\begin{definition}
définition comme classe de congruence de réels modulo $2\pi$ : la classe modulo $2\pi$ des arguments de $\overline z\cdot w$, qui est aussi celle des arguments de $\frac{w}{z}$.
\end{definition}


\begin{proposition}
Le cahier des charges est respecté, autrement dit : 
\begin{enumerate}
\item Chasles.
\item $\widehat{(\overrightarrow u, \overrightarrow u)} \equiv 0 \mod 2\pi$.
\item $\widehat{(\overrightarrow u, \overrightarrow v)} \equiv -\widehat{(\overrightarrow v, \overrightarrow u)} \mod 2\pi$.
\end{enumerate}
\end{proposition}

\begin{remarque}
Remarquer la ressemblance avec les propriétés du déterminant.
\end{remarque}