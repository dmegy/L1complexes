On suppose connues les fonctions trigonométriques $\cos$ et $\sin$, ainsi que l'exponentielle réelle $x\mapsto e^x$, ainsi que leurs propriétés, notamment les formules de trigonométrie pour $\cos(a+b)$ et $\sin(a+b)$, ainsi que les propriétés de l'exponentielle réelle : stricte croissance sur $\R$, jamais nulle, surjective sur $\R_+^*$.

\begin{mdframed}
Attention : la construction rigoureuse de ces fonctions n'a a priori pas encore été faite ! Il serait en théorie possible de la faire ici, mais il faudrait plusieurs chapitres du cours d'analyse, et ce serait contre-productif d'un point de vue pédagogique.

Le fait d'admettre l'existence et les propriétés classiques de ces fonctions ne crée pas de trou logique dans la suite. Tout ce qui a été admis pourra être démontré au second semestre sans utiliser le cours sur les nombres complexes, il n'y pas de cercle vicieux.
\end{mdframed}

\begin{definition}
Soit $z \in \C$, $a = Re(z)$ et $b = Im(z)$. Alors, on définit la fonction $exp : \C\to \C$ par $exp(z) = e^a (\cos(b)+i \sin(b))$. On écrit la plupart du temps $e^z$ au lieu de $exp(z)$, pour des raisons qui apparaitront plus bas.
\end{definition}

Tout de suite, quelques exemples:

\begin{exemples}
$e^0=1$, $e^{i\pi/2} = i$, $e^{i\pi}=-1$, $e^{3i\pi/2}=-i$ et $e^{2i\pi}=1$. 
\end{exemples}

\begin{mdframed}
En particulier l'exponentielle complexe n'est \textbf{pas injective} à la différence de l'exponentielle réelle.
\end{mdframed}

\begin{proposition}
L'exponentielle complexe vérifie les propriétés fondamentales suivantes:
\begin{enumerate}
\item L'exponentielle complexe prolonge l'exponentielle réelle.
\item $\forall z\in\C, |e^z| = e^{Re(z)}$, en particulier l'exponentielle complexe ne s'annule jamais.
\item $\forall (z,z')\in\C^2, e^{z+z'} = e^ze^{z'}$.
\item $\forall z\in\C, e^z=1 \iff z \in 2i\pi\Z$.
\end{enumerate}
\end{proposition}
\begin{proof}
Exercice. Découle de la définition et des formules usuelles pour l'exponentielle réelle et les fonctions trigonométriques, rappelées plus haut.
\end{proof}

L'exponentielle complexe vérifie également un certain nombre d'autres propriétés importantes, que l'on peut démontrer à l'aide de la définition ou directement à l'aide de la proposition précédente, sans utiliser la définition.

\begin{proposition}
\begin{enumerate}
\item $\forall z\in\C, \forall n\in \Z, \left(e^z\right)^n = e^{nz}$ (et on n'élève \underline{jamais} un complexe à une puissance non entière!);
\item $\forall z\in\C, e^{\overline{z}} = \overline{e^z}$;
\end{enumerate}
\end{proposition}

\begin{proof}
\begin{enumerate}
\item Récurrence immédiate en utilisant les propriétés précédentes.
\item Définition, ou simple conséquence de la proposition :  
\[ e^z\overline{e^z}=\abs{e^z}^2=\left(e^{\Re(z)}\right)^2 = e^{2\Re(z)}, \]
ce qui donne $\overline{e^z}=e^{2\Re(z)-z} = e^{\overline z}$.
\end{enumerate}
\end{proof}

\begin{proposition}
Soient $z$ et $z'$ des complexes. Alors
\[e^z = e^{z'} \Leftrightarrow \left(Re(z) = Re(z') \text{ et } Im(z)\equiv Im(z')\:[2\pi]\right).\]
\end{proposition}
\begin{proof}
On a :
\begin{align*}
e^z = e^{z'} & \Leftrightarrow e^{z-z'} = 1\\
 & \Leftrightarrow z-z' \in 2i\pi\Z\\
  & \Leftrightarrow \left( Re(z-z')=0 \text{ et } Im(z-z') \in 2\pi\Z\right)\\
  & \Leftrightarrow \left( Re(z) = Re(z') \text{ et } Im(z)\equiv Im(z')\:[2\pi]\right).
\end{align*}
\end{proof}


On a vu plus haut que l'exponentielle complexe n'est pas injective. Comme elle ne s'annule jamais, elle n'est pas surjective sur $\C$. Cependant, elle l'est en corestriction à $\C^*$, c'est-à-dire que tout complexe non nul possède un antécédent par l'exponentielle complexe:
\begin{proposition}
\[ \forall w\in\C^*, \exists z\in\C, e^z=w\]
\end{proposition}
\begin{proof}
Soit $w\in\C^*$. Notons $z=|w|$. Alors $w'=\frac{w}{r}\in\U$. Il existe\footnote{C'est ici que l'on utilise une propriété forte des fonctions cosinus et sinus que l'on n'a pas démontrée entièrement, parce qu'au fond on n'a jamais défini correctement le cosinus et le sinus. On peut choisir $\theta = \arccos(\Re(w'))$ ou son opposé mais c'est pareil, la construction de l'arccosinus n'a pas été faite en détail et dépend évidemment d'une définition rigoureuse du cosinus, de sa continuité etc.} alors $\theta\in\R$ tel que $w'=\cos(\theta)+i\sin(\theta)$. On en déduit que $w'=e^{i\theta}$.

Mais alors, on peut écrire $w=re^{i\theta}$. Pour finir, comme $r>0$, c'est l'exponentielle\footnote{Pareil, on utilise des propriétés admises sur l'exponentielle réelle et le logarithme qui n'ont pas été totalement démontrées.} d'un certain réel $l$, à savoir $l = \ln(r)$. Finalement, $w=e^le^{i\theta} = e^{l+i\theta}$. En posant $z=l+i\theta$, on a bien
\[ w = e^z.\]
\end{proof}

\begin{mdframed}
Pas plus qu'il n'existe de fonction \og racine carrée complexe\fg{} raisonnable, il n'existe de fonction \og logarithme complexe\fg. En tout cas, pas au sens des applications usuelles. Pour ceux qui continueront en maths jusqu'au M1 ou M2, vous apprendrez le fin mot de l'histoire avec les \emph{surfaces de Riemann}. Dans le cours d'analyse complexe de L3, vous commencerez à vous frotter de loin à ces objets, avec l'introduction de la \og détermination principale du logarithme complexe\fg, et de quelques chapitres de cours sur les \emph{revêtements}. Patience !

Jusqu'à ce moment, en aucun cas vous ne pouvez prendre le \og logarithme\fg{} d'un nombre complexe. Les notations $\log$ ou $\ln$ ne sont valides que devant un nombre réel strictement positif, la règle est toujours la même.
\end{mdframed}
