
Soit $z \in\C$ . On a 
\[z\overline z = \Re(z)^2 + \Im(z)^2.\]
Comme une somme de carrés de onmbres réels est un réel positif, on en déduit que $z\bar z$ est un réel positif. Il est donc licite de former la racine carrée de $z\overline z$.

\begin{definition}
On note $|z|$ et on appelle \emph{module de $z$} le nombre réel positif $\sqrt{z\overline z}$. 
\end{definition}


\begin{remarque}
Si $z\in \C$ et que $a+ib$ est sa forme algébrique, alors $z\bar z$ vaut $(a+ib)(a-ib)=a^2+b^2$, et donc 
\[ |z|=\sqrt{a^2+b^2}\]
Souvent, c'est cette expression qui est donnée comme définition du module, au lieu de $\sqrt{z\bar z}$.
\end{remarque}

En termes d'applications entre ensembles et non plus seulement d'éléments, le module est une application de $\C$ dans $\R_+$.

La notation avec les barres est déjà utilisée pour désigner la valeur absolue sur $\R$, mais si $z$ est réel, le module de $z$ est tout simplement la valeur absolue de $z$, il n'y a donc pas de conflit de notation\footnote{On dit que le module \emph{prolonge} la valeur absolue. Ou encore, que la restriction à $\R$ du module est la valeur absolue.}. 


\begin{exo}
Montrer que l'application \og module\fg{} de $\C$ dans $\R_+$, qui à $z$ associe $|z|$ est une application surjective, mais non injective.
\end{exo}
%\begin{proof}
%Surjectivité : soit $r\in\R_+$. Comme $\abs{r}=r$, on en déduit que l'application module est surjective sur $\R_+$. Pour la non-injectivité, on remarque que $\abs 1 = \abs i = \abs{-1}$.
%\end{proof}

\begin{proposition}
Le module est une application multiplicative, autrement dit:
\[ \forall (z,w)\in\C^2, \abs{z w}=\abs z \abs w.\]
\end{proposition}
\begin{proof}
On applique la définition. Soient $z$ et $w$ des complexes. Montrons que $\abs{z w}$ et $\abs z \abs w$ sont égaux. Comme ce sont des réels positifs, il revient au même de montrer que leurs carrés sont égaux. Or, on a par définition du module:
\[
\abs{zw}^2 
= zw\overline{zw} 
= z\overline z w\overline w 
= \abs{z}^2 \abs{w}^2 
= \left(\abs z\abs w\right)^2 
\]
\end{proof}

\begin{proposition}
Pour tout $z\in \C$, on a $\abs{\overline z} = \abs z$.
\end{proposition}
\begin{proof} Exercice.\end{proof}


\begin{proposition} 
\begin{enumerate}
\item $\forall z\in\C,\: |z|=0\iff z=0$.
\item $\forall z\in\C,\: \Re(z) \leq |z|$, avec égalité ssi $z\in\R_+$.
\item $\forall z,w\in\C,\: |z+w| \leq |z|+|w|$, avec égalité ssi $\overline z w\in\R_+$.
\end{enumerate}
\end{proposition}
\begin{proof}
\begin{enumerate}
\item Soit $z\in\C$. Comme $\abs z$ est un réel positif, il est nul ssi son carré est nul. On a donc:
\[ |z|=0\iff |z|^2=0\iff \Re(z)^2+\Im(z)^2=0 \iff \begin{cases}\Re(z)=0\\ \Im(z)=0\end{cases} \iff z=0.\]
\item On remarque que
\[\Re(z) \leq |\Re(z)| = \sqrt{\Re(z)^2} \leq \sqrt{\Re(z)^2+\Im(z)^2} = |z|\]
avec égalité ssi les deux inégalités sont des égalités, c'est-à-dire $\Im(z)=0$ et $\Re(z)=|\Re(z)|$, autrement dit $z\in\R_+$.
\item Il est équivalent de montrer l'inégalité entre les carrés des quantités, puisque celles-ci sont positives.
\[
|z+w|^2 
= |z|^2+2\Re(\overline{z}w) +|w|^2 
\leq |z|^2+2\abs{\overline z w}  +|w|^2
= |z|^2+2\abs z \abs w  +|w|^2
= (|z|+|w|)^2
\]
avec égalité ssi $\overline z w \in\R_+$ d'après le deuxième point.

\end{enumerate}
\end{proof}

\begin{definition} On appelle \emph{cercle unité de $\C$} et on note $\U$ l'ensemble 
\[ \U := \{z\in\C\:|\: |z|=1\}\]
\end{definition}

\begin{proposition}
\begin{enumerate}
\item $\U \subseteq \C^*$.
\item L'ensemble $\U$ est stable par produit, autrement dit si deux nombres complexes sont dans $\U_n$, alors leur produit aussi : $\forall (z,w)\in\U^2, zw\in\U$.
\item L'ensemble $\U$ est stable par inverse, autrement dit si un nombre complexe est dans $\U_n$, alors (il est inversible et) son inverse aussi : $\forall z\in\U, \frac{1}{z} \in\U$.
\end{enumerate}
\end{proposition}
\begin{remarque}
On résume ces deux propriétés en disant que $\U$ est un \emph{sous-groupe multiplicatif} de $\C^*$. Les notions générales de groupe et sous-groupe seront introduites par la suite, cette remarque sert juste à commencer à poser le vocabulaire.
\end{remarque}
\begin{exo}
Trouver des exemples de parties de $\C^*$ qui ne sont pas stables par produit, ou pas stables par inverse.
\end{exo}
\begin{proof}
\begin{enumerate}
\item Soit $z\in\U$. Comme $\abs z=1$, on a $z\neq 0$. Donc $z\in\C^*$.
\item Ceci découle de la multiplicativité du module. Soient $z$ et $w$ dans $\U$. Alors $\abs{zw} = \abs z \cdot \abs w = 1\times 1 = 1$, donc $zw\in\U$.
\item Soit $z\in\U$. On a $\abs{\frac1z} = \frac{1}{\abs z} = \frac11=1$.
\end{enumerate}
\end{proof}



Remarquons pour finir que l'application
\[ \theta : \begin{cases}\C^* \to \U,\\ z\mapsto \frac{z}{|z|}\end{cases}\]
est surjective et multiplicative.
