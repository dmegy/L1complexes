\documentclass[11pt,a4paper,oneside]{book}

\usepackage[top = 3cm, bottom = 3cm, left = 2cm, right = 7cm,marginpar=5cm,marginparsep=1cm]{geometry}% gestion des marges etc
\usepackage[utf8]{inputenc} % caractères utf8 dans le fichier source
\usepackage[T1]{fontenc} % encodage en sortie
\usepackage[french]{babel} % paramètres de langue : guillemets etc
\usepackage{amssymb,mathtools,amsthm}
\usepackage{stmaryrd,mathrsfs} % polices et symboles supplémentaires
\usepackage{mdframed,fancybox,graphicx}
\usepackage[dvipsnames]{xcolor}
\definecolor{preuve}{rgb}{0,0.2,0.5}

\usepackage[francais]{minitoc} % sommaires en début de chapitre
\setcounter{minitocdepth}{2} % profondeur des sommaires (1 = sections)
\setcounter{tocdepth}{1} % profondeur de la table des matières

\usepackage{fontawesome} % icônes
\usepackage{xypic,multicol,comment,variations,enumitem,datetime,microtype}
\usepackage{imakeidx}% avant hyperref sinon imakeidx ne met pas les liens?

\usepackage{pgf,pgfmath,tikz}
\usetikzlibrary{arrows}
\usetikzlibrary[patterns]
\tikzset{every picture/.style={execute at begin picture={
   \shorthandoff{:;!?};}
}}


\usepackage{marginfix} % floats sidenotes
\usepackage{environ,sidenotes}

\usepackage{hyperref}
\hypersetup{
    colorlinks=true,       % false: boxed links; true: colored links
    linkcolor=[rgb]{0.7,0.2,0.2},          % color of internal links
    citecolor=[rgb]{0.7,0.2,0.2},        % color of links to bibliography
    filecolor=[rgb]{0.7,0.2,0.2},      % color of file links
    urlcolor=[rgb]{0.7,0.2,0.2}           % color of external links
}



% - - - - - - -
% Spécifique à ce document :
%\usepackage{palatino, euler} % police : Palatino, et Euler pour les maths
\usepackage{fourier} % police de caractères : Adobe Utopia + Fourier math
\everymath{\displaystyle} % plus lisible mais casse l'homogénéité de la mise en page, tant pis la lisiblité passe en premier

\newcommand{\retourTOC}{Retour à la table des matières principale}

\newcommand{\N}{\mathbb{N}}
\newcommand{\Z}{\mathbb{Z}}
\newcommand{\Q}{\mathbb{Q}}
\newcommand{\R}{\mathbb{R}}
\newcommand{\C}{\mathbb{C}}
\newcommand{\K}{\mathbb{K}}
\renewcommand{\P}{\mathbb{P}}
\renewcommand{\S}{\mathbb{S}}
\newcommand{\B}{\mathbb{B}}
\newcommand{\U}{\mathbb{U}}
\DeclareMathOperator{\pgcd}{pgcd}
\DeclareMathOperator{\ppcm}{ppcm}
%\DeclareMathOperator{\exp}{exp}
\DeclareMathOperator{\Id}{Id}
\DeclareMathOperator{\Bij}{Bij}
\DeclareMathOperator{\Fix}{Fix}
\DeclareMathOperator{\Aff}{Aff} % affixe
\DeclareMathOperator{\Card}{Card} % cardinal
\DeclareMathOperator{\Coord}{Coord} % coordonnées
\DeclareMathOperator{\dist}{dist}
\renewcommand{\Re}{\operatorname{Ré}}
\renewcommand{\Im}{\operatorname{Im}}
\renewcommand{\mid}{\;\ifnum\currentgrouptype=16 \middle\fi|\;}
\newcommand\eqdef{\mathrel{\overset{\makebox[0pt]{\mbox{\normalfont\tiny\sffamily déf}}}{=}}}
% égal par définition

\newcommand{\ensemble}[2]{\left \{ #1  
    \ifx&#2&%
       %
    \else%
       \, \middle | \, #2%
    \fi%
\right \}}

\newcommand{\modulo}[1]{\:\left(\operatorname{mod}#1\right)}

\newcommand{\set}[1]{\left\{#1\right\}}
\newcommand{\abs}[1]{\left\lvert#1\right\rvert}


% Environnements : 
\swapnumbers
\theoremstyle{definition}
\newtheorem{theoreme}{Th\'eor\`eme}[section]
\newtheorem{proposition}[theoreme]{Proposition}
\newtheorem{corollaire}[theoreme]{Corollaire}
\newtheorem{lemme}[theoreme]{Lemme}
\renewenvironment{proof}{\color{preuve}\emph{Démonstration.~}}{\qed}
\newenvironment{red}{\begin{quote}\color{preuve}\emph{Exemple de rédaction:}\\}{\end{quote}}

\newtheorem{propdef}[theoreme]{Proposition et Définition}
\newtheorem{axiomedef}[theoreme]{Axiome et Définition}
\newtheorem{definition}[theoreme]{D\'efinition}
\newtheorem{vocabulaire}[theoreme]{Vocabulaire}

\newtheorem{exemple}[theoreme]{Exemple}
\newtheorem{exemples}[theoreme]{Exemples}
\newtheorem{attention}[theoreme]{Mise en garde}
\theoremstyle{plain}
%\newtheorem{rmq}[theoreme]{Remarque}
\newtheorem{exercice}[theoreme]{Exercice}% on laisse pour ne pas casser le code existant

\NewEnviron{remarque}{%
  \marginpar{\begin{mdframed}\footnotesize{\textbf{Remarque:} \BODY}\end{mdframed}}%
}
\NewEnviron{exo}{%
  \marginpar{\begin{mdframed}\footnotesize{\textbf{Exercice:} \BODY}\end{mdframed}}%
}
\NewEnviron{methode}{%
  \marginpar{\begin{mdframed}\footnotesize{\textbf{Méthode:} \BODY}\end{mdframed}}%
}



% 
% sidenotes en taille footnotesize :
% https://tex.stackexchange.com/questions/361622/changing-sidenote-size

\makeatletter
\RenewDocumentCommand\sidenotetext{ o o +m }{%      
    \IfNoValueOrEmptyTF{#1}{%
        \@sidenotes@placemarginal{#2}{\textsuperscript{\thesidenote}{}~\footnotesize#3}%
        \refstepcounter{sidenote}%
    }{%
        \@sidenotes@placemarginal{#2}{\textsuperscript{#1}~#3}%
    }%
}
\makeatother



\newcommand{\oldfootnote}{\footnote}
\renewcommand{\footnote}{\sidenote}





\dominitoc
\begin{document}
%%%%%%%%%%%%%%%%%%%%%%%%%%%%%%%%%%%%%%%%%%


\newgeometry{margin=2cm}

\begin{titlepage}


\includegraphics[width=4cm]{img/logo-IECL}
\hfill
\includegraphics[width=5cm]{img/logo-UL}\\
\vspace{3em}
\begin{center}
{\Huge Nombres complexes et géométrie, année 2020-2021}\\
\vspace{3em}
{Damien Mégy\\
\vspace{3em}
{\small \faGithub{}  Ce document et ses fichiers source sont disponibles à l'adresse\\
\url{http://github.com/dmegy/L1complexes}\\}
}

\vspace{3em}
Ce document est en cours de rédaction. Cette version est celle du \today{} à \currenttime{}. La version la plus récente est à l'adresse indiquée plus haut.\\
\begin{mdframed}
La page arche contient les feuilles de TD, les dernières avec corrections. Il y a également de nombreux (vieux) sujets de bac portant sur le programme de l'UE et qu'il est conseillé de regarder, en plus des feuilles de TD.
\end{mdframed}
\vspace{3em}
\end{center}
\end{titlepage}
%\maketitle

\addtocontents{toc}{\protect\hypertarget{toc}{}}
\tableofcontents
\restoregeometry

\section{Mode d'emploi et introduction}

\subsection{Comment lire ce texte}

Ce document ne prétend pas remplacer le cours au tableau, ni un livre de cours sur les nombres complexes et la géométrie plane. Il sert d'autres objectifs :


\begin{enumerate}
\item Avoir un texte de référence qui contient ce qui est au programme de cette UE précise cette année 2020-2012, ce qui n'est pas forcément le cas des livres à la BU (nouveaux programmes, réformes, évènements etc).
\item Insister sur certains points plus que ce qui est fait dans les ouvrages de référence, si les enseignants sentent que c'est nécessaire certaines années. 
\item Ajouter des remarques qui font le lien avec les autres cours, des conseils moins formels que dans un livre publié etc.
\end{enumerate}

La structure du texte est la suivante : 
\begin{itemize}
\item Dans la partie principale de la page, il y a le cours, sous forme relativement compacte : définitions, propositions, théorèmes, démonstrations et exemples (indispensables à la compréhension du reste).
\item Dans la marge latérale, assez importante, des exercices d'application directe du cours, des remarques, des notes, des digressions etc.
\end{itemize}

\begin{mdframed}
Ce qui est dans la marge n'est absolument pas \og moins important\fg, ou facultatif. Les exercices d'application, en particulier, sont prévus pour pouvoir être faits directement après la lecture du cours. En général, ils consistent simplement à appliquer la définition précédente, où à chercher un contre-exemple simple.

La disposition de ces éléments dans la marge permet simplement de conserver une structure assez concentrée pour le cours proprement dit, ce qui peut faciliter les révisions.
\end{mdframed}

\subsection{Notation \og $:=$\fg}

Dans tout ce document, le symbole \og $:=$\fg{} signifie \og par définition égal à \fg. Lorsque l'on écrit $A:=B$, cela signifie donc que $A$ est par définition égal à $B$. (Parfois, on écrira aussi $B=:A$  dans certains formules, lorsque le terme à définir doit être placé à droite et non à gauche pour des raisons esthétiques.)

Ce symbole n'est pas standard dans tous les textes mathématiques, mais il est très utile pour éviter des situations ambiguës.

\begin{remarque}
En informatique, l'égalité et l'affectation ont des symboles différents (en général, \og$==$\fg{} et \og$=$\fg). 
\end{remarque}

\chapter{Nombres complexes}


\section{Construction, partie réelle et imaginaire (en cours)}
\label{sec:construction}
Pour simplifier, on dit parfois que les nombres complexes sont obtenus à partir des nombres réels en \og ajoutant un élément $i$ qui vérifie la relation $i^2=-1$, et en calculant avec les règles de calcul usuelles\fg.

Cette phrase n'est pas vraiment satisfaisante : quel est cet élément ? Comment le fabrique-t-on ? Comment fait-on pour \og l'ajouter\fg ? Peut-on ajouter tout et n'importe quoi juste en le décrétant de la sorte ?

Pour bien faire sentir ce qu'une telle phrase a d'abusif, et en quoi elle ne peut pas être qualifiée de définition mathématique, considérons un court instant d'autres situations semblables : 
\begin{itemize}
\item Peut-on ajouter à $\R$ un élément \og$\infty$\fg{} qui soit plus grand que tous les nombres réels ? Peut-on continuer à calculer comme auparavant\footnote{Supposons que ce soit possible. Comment définir l'addition ? Si l'on décide que $\infty+2=\infty$, par exemple, comment gérer la soustraction ? On devrait par exemple pouvoir soustraire $\infty$ aux deux membres de l'équation précédente, mais cela conduirait à $2=0$... (Enfin, si on part du principe que $\infty-\infty=0$... mais est-ce le cas ? Est-ce quelque chose que l'on postule, ou bien quelque chose que l'on démontre ?) }?
\item Peut-on ajouter à $\R$ un élément $\alpha$ qui vérifie par exemple $\alpha\geq 2$ et aussi $\alpha\leq 1$ ? Peut-on continuer à calculer comme auparavant\footnote{Voir le cours sur les relations d'ordre (UE \og fondements des mathématiques\fg) : ce qui est sûr c'est que la relation obtenue ne peut pas être une relation d'ordre, car la transitivité impliquerait que $2\leq 1$, ce qui est faux. Donc même si on peut rajouter un tel élément, l'intérêt reste assez limité} ?
\end{itemize}

Si vous pensez que non, alors pourquoi accepter d'ajouter un élément $i$ vérifiant $i^2=-1$, propriété tout aussi impossible avec les nombres réels ?


Si on veut avoir un élément $i$ vérifiant $i^2=-1$ (relation impossible avec des nombres réels), il ne suffit pas d'affirmer qu'il existe et de dire qu'on le rajoute: il faut:
\begin{enumerate}
\item Le construire mathématiquement.
\item Ensuite il faut expliquer ce que signifie $i^2$ (c'est-à-dire $i\times i$). Il faut donc définir ce que signifie le produit dans ce nouveau contexte : comme $i$ n'est pas un nombre réel, la multiplication doit être définie. D'ailleurs, l'addition doit également être définie.
\item Ensuite, il faut montrer que les règles de calcul habituelles s'appliquent toujours avec ce produit et cette  somme sur les nouveaux objets que sont les nombres complexes (possibilité de soustraire, de diviser, distributivité etc). Un exemple tout simple : on a défini le produit de nombres complexes, et la somme de nombres complexes : mais a-t-on bien que $2\times i = i+i$, avec cette somme et ce produit ? Si on définit la somme et le produit n'importe comment, ça ne marchera pas.
\item Enfin, il faut démontrer qu'avec les nouvelles opérations $+$ et $\times$ que l'on a défini et qui se comportent comme on s'y attend, le produit $i\times i$ vaut effectivement $-1$.
\end{enumerate}    

Aucune de ces étapes ne va de soi.


\begin{mdframed}
Encore une fois, la morale est donc qu'on ne fabrique pas un objet mathématique juste en le voulant. Il faut construire concrètement les nouveaux objets, et démontrer mathématiquement qu'ils se comportent comme on le souhaite. Et parfois, ce n'est pas possible.

Heureusement, pour les nombres complexes, la construction est possible !
\end{mdframed}


Il existe plusieurs manières de construire les nombres complexes, qui ont toutes leur intérêt. La plus courante au niveau Terminale/L1 depuis une trentaine d'années\footnote{Auparavant, on utilisait des matrices, mais ça suppose de connaître le produit matriciel et d'être à l'aise en algèbre. Cette construction n'est plus adaptée aux programmes actuels. La méthode actuelle est plus simple et au bout du compte équivalente. Son seul défaut est de \og parachuter\fg{} la formule du produit sans trop expliquer d'où ça vient.} consiste à utiliser des couples de réels.

\paragraph{Construction}




\paragraph{Parties réelle et imaginaire, règles de calcul}

Dans ce paragraphe, on récapitule quelques propriétés de la partie réelle et de la partie imaginaire d'un nombre complexe. Si $z$ est un nombre complexe, on peut bien sûr considérer les nombres réels $\Re z$ et $\Im z$, mais de façon plus globale, la partie réelle et la partie imaginaire sont des applications:

\[ \Re : \quad \C\to \R, z\mapsto \Re(z)\]
\[ \Im : \quad \C\to \R, z\mapsto \Im(z)\]

%(Parfois, on considère ces deux applications comme étant à valeurs dans $\C$, en utilisant le plongement usuel (on dit \og canonique\fg) de $\R$ dans $\C$, qui permet de considérer tout nombre réel comme un nombre complexe.)

\begin{attention}
La partie imaginaire d'un nombre complexe est un \underline{réel}. La partie imaginaire de $1+2i$ est $2$, pas $2i$.
\end{attention}

\begin{exo}
L'application partie réelle $\Re : \C\to \R, z\mapsto \Re z$ est-elle injective ? Surjective ?
%\footnote{Réponses : non, et oui.}% attention, float dans un float -> erreur latex
\end{exo}

\begin{proposition}[Additivité]
Les applications $\Re$ et $\Im$ sont additives.
\end{proposition}
\begin{proof}
Exercice.
\end{proof}

L'additivité a la conséquence suivante, bien pratique pour certains calculs : 

\begin{corollaire}
Soit $n\in \N^*$ et $(z_k)_{1\leq k \leq n}$ une famille de nombres complexes. Alors, on a
\[ \Re\left(\sum_{k=1}^n z_k\right) = \sum_{k=1}^n \Re\left(z_k\right)\]
\[ \Im\left(\sum_{k=1}^n z_k \right)= \sum_{k=1}^n \Im\left(z_k\right)\]
Autrement dit, on peut \og sortir la somme \fg{} d'une partie réelle ou d'une partie imaginaire.
\end{corollaire}
\begin{proof}
Exercice. (Récurrence, en utilisant la proposition précédente (additivité) pour prouver l'hérédité.)
\end{proof}

\begin{proposition}[Homogénéité]
Les applications $\Re$ et $\Im$ sont homogènes, ce qui signifie la chose suivante:
\[ \forall z\in\C, \forall \lambda\in \R, \begin{cases}\Re(\lambda z)=\lambda \Re z\\ \Im(\lambda z) = \lambda \Im z\end{cases}\]
\end{proposition}

% éventuellement mettre en remarque l'homogénéité de degré supérieur ?
% par exemple https://fr.wikipedia.org/wiki/Fonction_homog%C3%A8ne

\begin{exemple}
On a par exemple $\Re(3z)=3\Re z$, $\Im\left(\frac{2z}{\sqrt 3}\right) = \frac{2}{\sqrt 3}\Im z$ ou encore $\Re(-\pi z)=-\pi\Re z$. (Exemples obtenus en prenant $\lambda=3$, $\lambda = \frac{2}{\sqrt 3}$ et $\lambda=-\pi$.)
\end{exemple}

\begin{attention}
Dans la propriété d'homogénéité, $\lambda$ doit être réel, et non complexe, sinon ça ne marche en général pas. C'est pour cela que l'on écrit parfois \og $\R$-homogène\fg{} au lieu de simplement \og homogène\fg.
\end{attention}

\begin{exo}
Montrer que l'assertion suivante est \textbf{fausse} :
\begin{multline*}
 \forall z\in \C, \forall \lambda \in \C,\\
  \Re(\lambda z)=\lambda \Re z.
\end{multline*}
\end{exo}

\begin{exo}
Montrer que $\Re$ et $\Im$ ne sont pas multiplicatives, c'est-à-dire que les deux assertions suivantes sont \textbf{fausses} :
\begin{multline*}
\forall (z,z')\in\C^2,\\ \Re(zz') = \Re(z)\cdot \Re(z')
\end{multline*}
et
\begin{multline*}
\forall (z,z')\in\C^2, \\ \Im(zz') = \Im(z)\cdot \Im(z')
\end{multline*}
\end{exo}

Le fait d'avoir à la fois l'homogénéité et l'additivité porte un nom spécial : la $\R$-linéarité. En général, cette propriété est formulée de la façon suivante.

\begin{proposition}
Les applications $\Re$ et $\Im$ sont $\R$-linéaires, ce qui signifie
\[ \forall z, w\in \C, \forall \lambda, \mu \in \R, \Re(\lambda z+\mu w) = \lambda \Re z + \mu \Re w\]
\[ \forall z, w\in \C, \forall \lambda, \mu \in \R, \Im(\lambda z+\mu w) = \lambda \Im z + \mu \Im w\]
\end{proposition}
\begin{proof}
Soient $z, w\in \C$ et $\lambda, \mu \in \R$. Alors on a:
\begin{align*}
\Re(\lambda z+\mu w) &= \Re(\lambda z) + \Re(\mu w) & \text{Additivité}\\
&= \lambda \Re z + \mu \Re w & \text{Homogénéité}
\end{align*}
La même preuve marche pour la partie imaginaire.
\end{proof}

\begin{mdframed}
Les propriétés d'additivité et de $\R$-linéarité sont exrêmement importantes, et vous les recroiserez à d'innombrables reprises les prochains mois (en particulier dans le cours d'algèbre linéaire). C'est une bonne chose de commencer à rencontrer ce mot assez tôt, et à apprendre tout doucement à utiliser la notion.
\end{mdframed}

La $\R$-linéairité a la conséquence suivante :

\begin{corollaire}
Soit $n\in \N^*$, $(z_k)_{1\leq k \leq n}$ une famille de nombres complexes et $(\lambda_k)_{1\leq k \leq n}$ une famille de nombres réels. Alors, on a
\[ \Re\left(\sum_{k=1}^n \lambda_kz_k\right) = \sum_{k=1}^n \lambda_k\Re\left(z_k\right)\]
\[ \Im\left(\sum_{k=1}^n \lambda_kz_k\right) = \sum_{k=1}^n \lambda_k\Im\left(z_k\right)\]
\end{corollaire}
\begin{proof}
Exercice. (Récurrence, en utilisant la proposition prédédente ($\R$-linéarité) pour prouver l'hérédité.)
\end{proof}



\section{Conjugaison}
\label{sec:conjugaison}

\begin{definition}
Soit $z\in\C$.
On appelle \emph{conjugué} de $z$ et on note $\overline z$ le nombre complexe $\Re(z) -i\Im(z)$.
\end{definition}

\begin{remarque}
Définition équivalente, plus lourde mais plus concrète : Soit $z\in \C$ et soient $a$ et $b$ les réels tels que $z=a+ib$. Alors $\overline z := a-ib$. 
\end{remarque}

\begin{exemples}
On a les égalités $\overline{5+3i}=5-3i$ et $\overline{-\sqrt 3-i\sqrt2}=-\sqrt 3+i\sqrt 2$. Remarquer aussi que $\overline 1=1$, $\overline 0 = 0$, et de façon générale, si $z$ est réel, $\overline z = z$.
\end{exemples}

\begin{definition}
La \emph{conjugaison complexe} est l'application de $\C$ dans $\C$, qui à un complexe $z$ associe son conjugué $\overline z$.
\end{definition}

\begin{proposition}[La conjugaison est involutive]
Appliquer deux fois successivement la conjugaison revient à ne rien faire du tout. En termes mathématiques:
\[ \forall z\in\C, \overline{(\overline z)} = z\]
On dit que la conjugaison est \emph{involutive}, ou bien que c'est une \emph{involution}.\footnote{De façon générale, si $f : E\to E$ est une application d'un ensemble dans lui-même, on dit que $f$ est involutive si $f\circ f=\Id_E$.}
\end{proposition}

Par exemple, si $z=-2+5i$, on a $\overline z = \overline{-2+5i} = -2-5i$, et 
\[ \overline{\overline z} = \overline{-2-5i} = -2+5i = z.\]

\begin{proposition}
La conjugaison complexe est bijective.
\end{proposition}
\begin{proof}
On peut le montrer directement, mais on peut le déduire de la proposition précédente\footnote{Cette preuve marche dans une situation plus générale : toute application involutive est automatiquement bijective, car elle est sa propre application réciproque.}. 
\begin{enumerate}
\item Preuve d'injectivité. Soient $z$ et $w$ des complexes tels que $\overline z = \overline w$. En conjugant une fois de plus on obtient $\overline{\overline z}=\overline{\overline w}$ c'est-à-dire par involutivité $z=w$. Ceci montre que la conjugaison est injective.
\item Preuve de surjectivité. Soit $z\in\C$. D'après la propriété d'involutivité, on a $\overline{\left(\overline z\right)}=z$. Ceci montre que la conjugaison est surjective.
\end{enumerate}
\end{proof}


Les règles de calcul utilisables lorsque l'on manipule la conjugaison sont résumés dans la proposition suivante : 

\begin{proposition}[La conjugaison complexe est un automorphisme de corps]
\begin{enumerate}
\item La conjugaison est additive, ce qui signifie $\forall (z, z')\in \C^2, \overline{z+z'} = \overline{z}+\overline{z'}$. 
\item La conjugaison est multiplicative, ce qui signifie $\forall (z,z')\in\C^2, \overline{zz'} = \overline{z}\cdot\overline{z'}$.
\end{enumerate}
On résume ces deux propriétés en disant que \og la conjugaison est un \emph{(auto)morphisme de corps}.\fg{}
% 
\end{proposition}

\begin{remarque}
Comparatif entre les deux rédactions d'additivité : la première rédaction nécessite d'introduire quatre nouveaux symboles ($a$, $a'$, $b$ et $b'$), ensuite il y a juste des calculs. La deuxième rédaction ne nécessite pas d'introduire des notations, elle n'utilise que la définition générale avec les parties réelles et imaginaires. Elle ne nécessite même pas de savoir exactement ce que sont les parties réelles et imaginaires, simplement de savoir qu'elles sont elles-mêmes additives. Remarquer aussi que les points communs entre la première démonstration et la démonstration d'additivité pour $\Re$ et $\Im$ à la section précédente : de fait, on refait un peu le même travail en double.
\end{remarque}

\begin{proof}
\begin{enumerate}
\item Voici deux rédactions possibles de l'additivité, l'une directe et un peu \og terre-à-terre\fg, l'autre réutilisant l'additivité de $\Re$ et $\Im$, qui a déjà été montrée plus haut.
\begin{enumerate}
\item Soient $z$ et $z'$ des nombres complexes, et soient $a+ib$ et $a'+ib'$ leur écriture cartésienne. Alors, on a 
\begin{align*}
\overline{z+z'} &= \overline{a+ib+a'+ib'}\\
&= \overline{a+a'+i(b+b')} &\text{(Regroupement de termes)}\\
&=a+a'-i(b+b')& \text{(Définition de la conjugaison)}\\
&=(a-ib)+(a'-ib') & \text{(Regroupement de termes)}\\
&= \overline{z}+\overline{z'}.
\end{align*}
\item Deuxième rédaction, un peu plus abstraite. Soient $z$ et $z'$ des nombres complexes. Alors, on a
\begin{align*}
\overline{z+z'} &= \Re(z+z')-i\Im(z+z') &\text{(Définition de la conjugaison)}\\
&= \Re z+\Re z' -i\left(\Im z+\Im z'\right) & \text{(Additivité de $\Re$ et $\Im$)}\\
&=\left(\Re z- i\Im z\right) + \left(\Re z'- i\Im z'\right) & \text{(Regroupement de termes)}\\
&= \overline{z} + \overline{z'}. & \text{(Définition de la conjugaison)}
\end{align*}
\end{enumerate}

\item Multiplicativité. Soient $z$ et $z'$ des nombres complexes, et soient $a+ib$ et $a'+ib'$ leur écriture cartésienne. Alors, d'une part on a :
\begin{align*}
\overline{zz'} &= \overline{(a+ib)(a'+ib')}\\
&=\overline{aa'-bb'+i(ab'+ba')} & \text{(Calcul)}\\
&=aa'-bb'-i(ab'+ba')& \text{(Définition de la conjugaison)}\\
\end{align*}
Et d'autre part, on a 
\begin{align*}
\overline{z}\cdot \overline{z'} &= \overline{a+ib}\cdot \overline{a'+ib'}\\
&=(a-ib)(a'-ib')& \text{(Définition de la conjugaison)}\\
&=aa'-bb'-i(ab'+ba') & \text{(Calcul)}
\end{align*}
On en déduit que l'on a bien $\overline{zz'} = \overline{z}\cdot\overline{z'}$.
\end{enumerate}
\end{proof}



\begin{remarque}
Les propriétés d'additivité et de multiplicativité sont parfois invoquées par les slogans \og la conjugaison est compatible à la somme\fg{} ou \og le conjugué de la somme est égal à la somme des conjugués\fg{} pour l'additivité, et \og la conjugaison est compatible au produit\fg{} ou \og le conjugué d'un produit est égal au produit des conjugués\fg{} pour la multiplicativité. Cela dit, le terme \og compatible\fg{} n'est pas ce qu'il y a de plus précis, il vaut mieux l'éviter. Dans toute la suite, on utilisera les termes \og additif\fg{} et \og multiplicatif\fg, et \og automorphisme de corps\fg{} pour la conjonction des deux.
\end{remarque}

Comme conséquence immédiate, nous avons la

\begin{proposition}[Conjuguaison d'inverses et puissances]
\begin{enumerate}
\item Soit $z\in\C^*$. Alors $\overline{\left(\frac{1}{z}\right)} = \frac{1}{\overline z}$.
\item Soit $z\in\C$. Alors $\forall n\in\N, \overline{z}^n = \overline{z^n}$.
\end{enumerate}
\end{proposition}

\begin{proof}
\begin{enumerate}
\item On utilise la multiplicativité de la conjugaison. Comme $\frac1z\cdot z=1$, en prenant les conjugués on obtient
\[ \overline{\frac1z\cdot z} = \overline{\left(\frac1z\right)}\cdot \overline{z} = \overline 1 = 1,\]
d'où on déduit que \[ \frac{1}{\overline z} = \overline{\left(\frac1z\right)}.\]
\item Même si la proposition a l'air évidente, il faut la démontrer. On le fait par récurrence. Pour tou entier naturel $n$, notons $A(n)$ l'assertion \og $\overline{z}^n = \overline{z^n}$\fg. Comme $\overline{z}^0=1=\overline{z^0}$, l'assertion $A(0)$ est vraie. Soit maintenant $n\in\N$, et supposons que $A(n)$ soit vraie. Alors on a 
\begin{align*}
\overline{z}^{n+1}
&=\overline{z}^{n}\cdot\overline{z} \quad \text{par multiplicativité}\\
&=\overline{z^n}\cdot \overline{z} \quad \text{par hypothèse de récurrence}\\
&=\overline{z^n\cdot z} \quad \text{par multiplicativité}\\
&=\overline{z^{n+1}}
\end{align*}
Donc $A(n+1)$ est vraie, ce qui conclut d'après le principe de récurrence.
\end{enumerate}
\end{proof}


\begin{proposition}
Soient $z, z' \in \C, \lambda\in\R$. Alors
\[z\in\R \Leftrightarrow z=\overline z,\quad z\in i\R \Leftrightarrow z=-\overline{z},\]
\[\Re(z) = \frac{z+\overline z}{2},\quad \Im(z) = \frac{z-\overline z}{2i}.\]
\end{proposition}



\section{Module}
\label{sec:module}

Soit $z \in\C$ . On a 
\[z\overline z = \Re(z)^2 + \Im(z)^2 \in \R_+.\]
Il est donc licite de former la racine carrée de $z\overline z$, qui est un réel positif.

\begin{definition}
On note $|z|$ et on appelle \emph{module de $z$} le nombre réel positif $\sqrt{z\overline z}$. 
\end{definition}

Si $z$ est réel, le module de $z$ est la valeur absolue de $z$. Ceci explique la notation $|z|$ pour le module. En termes d'applications entre ensembles et non plus seulement d'éléments, le module est une application de $\C$ dans $\R_+$.

\begin{proposition}
Le module, vu comme application de $\C$ dans $\R_+$, est une application surjective, mais non injective.
\end{proposition}
\begin{proof}
Surjectivité : soit $r\in\R_+$. Comme $\abs{r}=r$, on en déduit que l'application module est surjective sur $\R_+$. Pour la non-injectivité, on remarque que $\abs 1 = \abs i = \abs{-1}$.
\end{proof}

\begin{proposition}
Le module est une application multiplicative, autrement dit:
\[ \forall (z,w)\in\C^2, \abs{z w}=\abs z \abs w.\]
\end{proposition}
\begin{proof}
On applique la définition. Soient $z$ et $w$ des complexes. Montrons que $\abs{z w}$ et $\abs z \abs w$ sont égaux. Comme ce sont des réels positifs, il revient au même de montrer que leurs carrés sont égaux. Or, on a par définition du module:
\[
\abs{zw}^2 
= zw\overline{zw} 
= z\overline z w\overline w 
= \abs{z}^2 \abs{w}^2 
= \left(\abs z\abs w\right)^2 
\]
\end{proof}

\begin{proposition}
Pour tout $z\in \C$, on a $\abs{\overline z} = \abs z$.
\end{proposition}
\begin{proof} Exercice.\end{proof}


\begin{proposition} 
\begin{enumerate}
\item $\forall z\in\C,\: |z|=0\iff z=0$.
\item $\forall z\in\C,\: \Re(z) \leq |z|$, avec égalité ssi $z\in\R_+$.
\item $\forall z,w\in\C,\: |z+w| \leq |z|+|w|$, avec égalité ssi $\overline z w\in\R_+$.
\end{enumerate}
\end{proposition}
\begin{proof}
\begin{enumerate}
\item Soit $z\in\C$. Comme $\abs z$ est un réel positif, il est nul ssi son carré est nul. On a donc:
\[ |z|=0\iff |z|^2=0\iff \Re(z)^2+\Im(z)^2=0 \iff \begin{cases}\Re(z)=0\\ \Im(z)=0\end{cases} \iff z=0.\]
\item On remarque que
\[\Re(z) \leq |\Re(z)| = \sqrt{\Re(z)^2} \leq \sqrt{\Re(z)^2+\Im(z)^2} = |z|\]
avec égalité ssi les deux inégalités sont des égalités, c'est-à-dire $\Im(z)=0$ et $\Re(z)=|\Re(z)|$, autrement dit $z\in\R_+$.
\item Il est équivalent de montrer l'inégalité entre les carrés des quantités, puisque celles-ci sont positives.
\[
|z+w|^2 
= |z|^2+2\Re(\overline{z}w) +|w|^2 
\leq |z|^2+2\abs{\overline z w}  +|w|^2
= |z|^2+2\abs z \abs w  +|w|^2
= (|z|+|w|)^2
\]
avec égalité ssi $\overline z w \in\R_+$ d'après le deuxième point.

\end{enumerate}
\end{proof}

\begin{definition} On appelle \emph{cercle unité de $\C$} et on note $\U$ l'ensemble $\{z\in\C\:|\: |z|=1\}$.
\end{definition}

\begin{proposition}
\begin{enumerate}
\item $\U \subseteq \C^*$.
\item Stabilité de $\U$ par produit : $\forall (z,w)\in\U^2, zw\in\U$.
\item Stabilité de $\U$ par inverse : $\forall z\in\U, \frac{1}{z} \in\U$.
\end{enumerate}
On résume ces deux propriétés en disant que $\U$ est un \emph{sous-groupe multiplicatif} de $\C^*$.
\end{proposition}
\begin{proof}
\begin{enumerate}
\item Soit $z\in\U$. Comme $\abs z=1$, on a $z\neq 0$. Donc $z\in\C^*$.
\item Ceci découle de la multiplicativité du module. Soient $z$ et $w$ dans $\U$. Alors $\abs{zw} = \abs z \cdot \abs w = 1\times 1 = 1$, donc $zw\in\U$.
\item Soit $z\in\U$. On a $\abs{\frac1z} = \frac{1}{\abs z} = \frac11=1$.
\end{enumerate}
\end{proof}



Remarquons pour finir que l'application
\[ \theta : \begin{cases}\C^* \to \U,\\ z\mapsto \frac{z}{|z|}\end{cases}\]
est surjective et multiplicative.


\section{Équation du second degré (en cours)}
\label{sec:second_degre}
\begin{proposition}
Soit $\alpha\in\C$.
Considérons l'équation $z^2=\alpha$, d'inconnue $z\in\C$.
\begin{enumerate}
\item Si $\alpha=0$, l'équation admet une unique solution, la solution nulle.
\item Si $\alpha\neq 0$, l'équation admet deux solutions distinctes, appelées \emph{racines carrées complexes\footnote{Ne PAS utiliser le symbole $\sqrt{\phantom{aa}}$, voir avertissement plus bas.} de $\alpha$}.
\end{enumerate}
\end{proposition}

\begin{proof}
\begin{enumerate}
\item Soit $z\in \C$. Si $z^2=0$, alors $z=0$. (Peut sembler évident mais attention aux pièges\footnote{On ne peut évidemment pas jutifier cela en \og passant à la racine carrée\fg{} come on le ferait sur $\R$ : il n'y a pas d'application racine carrée définie sur les nombres complexes, voir avertissement plus bas}... Preuve 1 : on a $z\times z=0$ donc un des deux facteurs est nul, et donc $z=0$. Preuve 2 : par l'absurde, si $z$ était non nul, il aurait un inverse noté $z^{-1}$, on pourrait alors multiplier l'équation $z^2=0$ par $z^{-2}$ et on obtiendrait $0=1$, absurde. Preuve 3 : si $z^2=0$, alors en prenant le module $0=\abs{z^2}=|z|^2$ donc\footnote{Si vous flairez un cercle vicieux ici c'est bien, vous êtes attentifs. Mais on admet que le résultat sur $\R$ est, lui, connu.} $|z|=0$ donc $z=0$.)
\item Si $\alpha\neq 0$, la situation est un peu plus délicate. Soit $z\in \C$ et soient $x$ et $y$ ses parties réelles et imaginaires. Notons également $a$ et $b$ les parties réelles et imaginaires de $\alpha$, qui vérifient donc $(a,b)\neq (0,0)$. On a 
\[ z^2=\alpha \iff x^2-y^2+2ixy=a+ib\]
En identifiant les parties réelles et imaginaires, on obtient
\[ z^2=\alpha \iff \begin{cases}x^2-y^2 &= a \\ 2xy&=b\end{cases}\]
Ce système de deux équations à deux inconnues $x$ et $y$ n'est pas linéaire en $x$ et $y$ et  n'est donc pas trivial à résoudre. Il est préférable d'exploiter en plus l'égalité des modules des deux membres de l'équation complexe. 
En prenant en effet le module  des membres de $z^2=\alpha$, on obtient $\abs{z^2}=\abs{\alpha}$ c'est-à-dire $x^2+y^2=|\alpha|$. Finalement, nous avons donc l'équivalence :
\[ z^2=\alpha \iff \begin{cases}x^2+y^2 &= |\alpha| \\ x^2-y^2 &= a \\ 2xy &= b\end{cases}\]
Les deux premières équations de ce système sont liénaires en $x^2$ et $y^2$ et ce système admet une solution unique pour le couple $(x^2,y^2)$, ce qui donne potentiellement jusqu'à quatre solutions possibles pour le couple $(x,y)$.

La dernière équation $2xy=b$ permet alors de ne garder que deux solutions pour le couple $(x,y)$, car elle fixe le signe de $xy$.
\end{enumerate}
\end{proof}

On ne donne pas la formule générale à dessein : d'une part il est inutile de la retenir, d'autre part il faut avant tout s'exercer sur des exemples, plus ou moins simples. Ceux faits en cours et TD sont simples, mais il est recommandé de résoudre ensuite l'équation $z^2=1+i$, ce qui donne des radicaux imbriqués.\footnote{Evidemment, si au cours d'un calcul on obtient des expressions vraiment très compliquées, le premier réflexe doit être de vérifier les calculs précédents.}


\begin{attention}
On n'utilise pas le symbole $\sqrt{\phantom{aa}}$ pour écrire les racines carrées complexes. Jusqu'à nouvel ordre, le symbole $\sqrt{\phantom{aa}}$ ne peut s'utiliser que sur un \textbf{réel positif}. Toute mauvaise utilisation de ce symbole entraînera une perte très importante de points aux examens.
\end{attention}

\section{Congruences}
\label{sec:congruences}
\begin{definition}
Soit $\alpha\in\R$. On note $\alpha\Z$ l'ensemble
\[ \alpha\Z=\ensemble{k\alpha}{k\in\Z}\]
C'est un sous-ensemble de $\R$.
\end{definition}

\begin{exemples}
\begin{enumerate}
\item L'ensemble $3\Z$ est $\set{\cdots -3, 0, 3, 6, 9, \cdots}$.
\item L'ensemble $\sqrt3\Z$ est $\set{\cdots -2\sqrt 3, -\sqrt 3, 0, \sqrt 3, 2\sqrt 3, 3\sqrt 3,\cdots}$.
\item L'ensemble $2\pi\Z$ est $\set{\cdots -6\pi, -4\pi, -2\pi, 0, 2\pi, \cdots}$.
\item L'ensemble $0\Z$ est $\{\cdots, -2\times0, -1\times 0, 0, 1\times 0, 2\times 0, \cdots\}$ c'est-à-dire tout simplement $\{0\}$.
\end{enumerate}
\end{exemples}

On voit donc qu'un ensemble du type $\alpha\Z$ est soit infini, soit $\{0\}$ dans le cas spécial où $\alpha=0$. Dans la suite, on n'utilisera les ensembles du type $\alpha\Z$ que lorsque $\alpha\neq$. \\

\fbox{On fixe donc un nombre réel $\alpha\neq 0$ pour toute la suite.}




\begin{proposition}
Soient $x$ et $y$ réels, et $a\neq 0$ un réel non nul. Les assertions suivantes sont équivalentes:
\begin{enumerate}
\item $\frac{x-y}{\alpha} \in \Z$;
\item $x-y \in \alpha\Z$;
\item $(\exists k\in\Z,\: x-y=k\alpha)$;
\item $(\exists k\in\Z,\: x= y + k\alpha)$.
\end{enumerate}
\end{proposition}
\begin{proof}
Exercice. Remarquer que si $t$ est un réel, l'assertion \og $t\in\Z$\fg{} est équivalente à \og$\exists k\in\Z, t=k$\fg.
\end{proof}


\begin{attention}
Certains ont pris l'habitude d'écrire des choses comme:
\[ x=y+2k\pi, k\in\Z.\]
Il faut se \textbf{débarasser} de cette mauvaise habitude, qui a peut-être été pardonnée les années précédentes, comme beaucoup d'autres choses, mais qui ne le sera plus. En pratique, ceux qui écrivent de cette façon peuvent rarement expliquer si leur égalité est vraie pour tout $k$, pour certains, pour un seul, la rédaction est floue et d'ailleurs c'est souvent un peu l'objectif... Cette faute sera fortement sanctionnée jusqu'à ce que la mauvaise habitude disparaisse. Soit la variable $k$ a été déclarée au préalable, soit c'est $\forall k\in\Z, x=y+2k\pi$, soit c'est $\exists k\in\Z, x=y+2k\pi$
\end{attention}

\begin{definition}
Soient $x$ et $y$ des réels.
On dit que $x$ est congru à $y$ modulo $\alpha$, et on écrit $x\equiv y~[\alpha]$, ou bien $x\equiv y \pmod \alpha$, si une des conditions équivalentes ci-dessus est vérifiée, autrement dit si le réel $x-y$ est un multiple \textbf{entier} (positif ou négatif) de $\alpha$.
\end{definition}

\begin{remarque}
La première assertion $\frac{x-y}{\alpha} \in\Z$ est souvent la plus maniable. C'est en général celle-là que l'on utilise dans les démonstrations.
\end{remarque}

\begin{attention}
Certains ont commencé à étudier les congruences au lycée, souvent en arithmétique. C'est une notion assez subtile: certaines règles intuitives s'appliquent, d'autres non. De plus, on étudie ici une version générale (congruence modulo un réel $\alpha$), et certaines propriétés vraies dans les cas simples deviennent fausses en général. Par exemple, on ne peut \textbf{pas} multiplier les congruences en général, contrairement à une idée reçue tenace. Voir plus bas. 
\end{attention}

\begin{exemples}
\begin{enumerate}
\item $1 \equiv 5~[2]$, car $1-5 = -4$ est un multiple de $2$.
\item $4\equiv -9\sqrt{3}+4~[\sqrt{3}]$, car $4 - (-9\sqrt{3}+4) = 9\sqrt{3}$ est un multiple de $\sqrt{3}$.
\item $\pi/3 \equiv 7\pi/3~[2\pi]$, car $\pi/3 - 13\pi/3 = -12\pi/3 = -4\pi$ est un multiple de $2\pi$.
\end{enumerate}
\end{exemples}



\begin{proposition}[La congruence est une relation d'équivalence]
La congruence vérifie les trois propriétés suivantes : 
\begin{enumerate}
\item Réflexivité : $\forall x\in\R, x\equiv x\mod \alpha$;\\
\item Symétrie : $\forall (x,y)\in\R^2, (x\equiv y \mod \alpha)\implies (y\equiv x \mod \alpha)$;\\
\item Transitivité : $\forall (x,y,z)\in\R^3, (x\equiv y\mod \alpha\text{ et }y\equiv z\mod \alpha)\implies (x\equiv z\mod \alpha)$.
\end{enumerate}
On résume ces trois propriétés en disant que la congruence modulo $\alpha$ est une relation d'équivalence. 
 \end{proposition}
 \begin{proof}
Exercice.
 \end{proof}


\begin{proposition}[addition et multiplications de congruences]
Soit $b\neq 0$ un réel non nul et $x$, $y$, $x'$, $y'$ des réels.
\begin{itemize}
\item Addition : si $x \equiv y\mod \alpha$ et $x' \equiv y'\mod \alpha$, alors: $x+x' \equiv y+y'\mod \alpha$.
\item Multiplication : on a l'équivalence $x \equiv y\mod \alpha \iff bx\equiv by \mod{b\alpha}$. 
\end{itemize}
\end{proposition}
\begin{proof}
\begin{enumerate}
\item Par la proposition précédente, on a $\frac{x-y}{\alpha} \in \Z$ et $\frac{x'-y'}{\alpha} \in \Z$. Donc $\frac{x-y}{\alpha}+\frac{x'-y'}{\alpha} \in \Z$, c'est-à-dire  $\frac{(x+x')-(y+y')}{\alpha} \in \Z$, c'est-à-dire $x+x' \equiv y+y' \mod \alpha$.
\item On a une chaîne d'équivalences : 
\[\left(x\equiv y\mod \alpha\right) \Leftrightarrow  \left(\frac{x-y}{\alpha} \in \Z\right) \Leftrightarrow \left(\frac{bx-by}{b\alpha} \in \Z\right) \Leftrightarrow \left(bx\equiv by\mod b\alpha\right).\]
\end{enumerate}
\end{proof}

\begin{attention}
\begin{enumerate}
\item Pour l'addition, ce n'est pas une équivalence, au sens où si on sait seulement que $x+x' \equiv y+y'\mod \alpha$, on ne peut en aucun cas conclure que $x \equiv y\mod \alpha$ et $x' \equiv y'\mod \alpha$ ! On a par exemple $1+3\equiv 0+0 \mod 4$, mais $1\not\equiv 0 \mod 4$ et $3\not\equiv 0 \mod 4$. 
\item Pour la multiplication, ne pas oublier de multiplier également la base de congruence, c'est-à-dire de remplacer le modulo $\alpha$ par modulo $b\alpha$ ! 
\end{enumerate}
\end{attention}

\begin{attention}
Il est important d'insister sur le dernier point : \textbf{on ne peut pas multiplier une congruence par un facteur sans multiplier également la base de congruence}. Il est facile d'exhiber des contre-exemples:

\begin{align*}
6\equiv 4 \mod 2 &\text{, et pourtant } 6\pi \not\equiv 4\pi \mod 2\\
6\equiv 4 \mod 2 &\text{, et pourtant } 6\sqrt2 \not\equiv 4\sqrt2 \mod 2\\
6\equiv 4 \mod 2 &\text{, et pourtant } 6\cdot \frac12 \not\equiv 4\cdot\frac12 \mod 2\\
\end{align*}

On ne peut pas non plus multiplier deux congruences entre elles, même si la base de congruence est identique:
\[ \begin{cases} 1\equiv 0 \mod 1\\ \sqrt 3 \equiv 1+\sqrt 3 \mod 1\end{cases}, \text{ et pourtant }
\sqrt 3 \not\equiv 0 \mod 1
\]
\end{attention}

\begin{comment}
Les erreurs de compréhension sur la multiplication semblent être causées par le fait suivant:
\begin{exercice}
Soient $x$ et $y$ des réels, et $n$ un \underline{entier} (relatif). Si $x\equiv y\mod \alpha$, alors $nx\equiv ny \mod \alpha$.
\end{exercice}
Ce résultat est correct et sa preuve est facile, mais :
\begin{enumerate}
\item Il est moins précis que la proposition précédente, qui affirme $nx\equiv ny \mod n\alpha$. Pr exemple, si $x\equiv y \mod 3$, il est exact que $2x\equiv 2y \mod 3$, mais ceci est moins précis que $2x\equiv 2y \mod 6$.
\item Ce n'est pas une équivalence. Par exemple, même si $2x\equiv 2y \mod 3$, on ne peut en aucun cas en déduire que $x\equiv y \mod 3$.
\end{enumerate}
Donc même dans ce cas, il est fortement déconseillé d'utiliser cet énoncé : même dans les cas où il est correct, il fournit un outil moins fort et moins maniable que la proposition.
\end{comment}

Les règles de calcul énoncées dans la proposition servent couramment à résoudre des équations faisant intervenir des congruences :
\begin{exercice}
Résoudre sur $\R$ l'équation $2x+5\equiv \sqrt 3 \mod 7$.
\begin{red}
Soit $x\in \R$. Alors on a 
\begin{align*}
2x+5\equiv \sqrt 3 \mod 7 
&\iff 2x\equiv \sqrt 3 - 5 \mod 7  & \text{(addition de congruences)}\\
&\iff x\equiv \frac{\sqrt3-5}{2} \mod{\frac72} &\text{(multiplication par $1/2$)}
\end{align*}
\end{red}
(Écrire $7$ au lieu de $7/2$ à la fin revient à oublier la moitié des solutions.)
\end{exercice}
Autre exemple:
\begin{exercice}
Résoudre sur $\R$ l'équation $\cos(2x+1)=\sin(x)$.
\begin{red}
Soit $x\in \R$. Alors on a 
\begin{align*}
\cos(2x+1)=\sin(x) 
&\iff \cos(2x+1)=\cos(\pi/2-x)  &  & \\
&\iff 2x+1\equiv \frac{\pi}{2}-x \mod 2\pi &\text{ OU }&  2x+1\equiv x-\frac{\pi}{2} \mod 2\pi   \\
&\iff 3x\equiv \frac{\pi}{2}-1 \mod 2\pi &\text{ OU }&  x\equiv -1-\frac{\pi}{2} \mod 2\pi   \\
&\iff x\equiv \frac{\pi}{6}-\frac13 \mod{\frac{2\pi}{3}} &\text{ OU }&  x\equiv -1-\frac{\pi}{2} \mod 2\pi 
\end{align*}
\end{red}
\end{exercice}


On termine par deux résultats qui servent très souvent.

\begin{exemple}[très important pour la suite]
Soit $n \in \N^*$. L'équation 
\[n\theta \equiv 0\mod2\pi,\]
d'inconnue $\theta \in [0,2\pi[$, admet $n$ solutions. En effet, on a 
\[n\theta \equiv 0\mod 2\pi \iff \theta \equiv 0\mod 2\pi/n \iff \left(\exists k \in \llbracket 0,n-1 \rrbracket, \theta = \frac{2k\pi}{n}\right).\]
\end{exemple}




\begin{proposition}
Soit $a\in\R_+^*$ et $x \in \R$.
Alors il existe un unique réel $y \in[0,a[ $ tel que $x \equiv y\quad [a]$.
\end{proposition}

\begin{proof}
Soit $y \in [0,a[$. On a donc $0\leq \frac{y}{a} < 1$, donc $-\frac{y}{a}\leq 0 < 1 -\frac{y}{a}$. En ajoutant $\frac{x}{a}$ aux trois membres on obtient: $\left(\frac{x-y}{a}\right)\leq \frac{x}{a} < 1 +\left(\frac{x-y}{a}\right)$.
On en déduit les équivalences suivantes:
\[ \frac{x-y}{a} \in \Z \Leftrightarrow \frac{x-y}{a} = \left\lfloor \frac{x}{a}\right\rfloor \Leftrightarrow y=x-a.\left\lfloor\frac{x}{a}\right\rfloor.\]
Ceci montre qu'il existe un unique réel $y \in [0,a[$ tel que $ \frac{x-y}{a} \in \Z$, c'est-à-dire tel que $x \equiv y\quad [a]$. Il est donné par $y=x-a.\left\lfloor\frac{x}{a}\right\rfloor$.
\end{proof}

\paragraph{Classes de congruence}

\begin{definition}
Soit $x$ un réel. On appelle \textbf{classe de congruence de $x$ modulo $\alpha$} et on note $[x]_\alpha$  l'ensemble des réels qui sont congruents à $x$ modulo $\alpha$, autrement dit c'est l'ensemble : 
\[ [x]_\alpha=\ensemble{y\in\R}{y\equiv x \mod \alpha}\]
D'après l'équivalence de définitions prouvée plus haut, on peut écrire cet ensemble sous la forme
\[ [x]_\alpha=\ensemble {x+k\alpha}{k\in\Z}\]
% Pour cette raison, on note également la classe de congruence sous la forme suivante : 
% \[ [x]_\alpha=x+\alpha\Z\]
% trop tôt, nécessiterait trop de mises en garde sur la notation
\end{definition}


\section{Exponentielle complexe et argument (en cours)}
\label{sec:exp}
On suppose connues les fonctions trigonométriques $\cos$ et $\sin$, ainsi que l'exponentielle réelle $x\mapsto e^x$, ainsi que leurs propriétés, notamment les formules de trigonométrie pour $\cos(a+b)$ et $\sin(a+b)$, ainsi que les propriétés de l'exponentielle réelle : stricte croissance sur $\R$, jamais nulle, surjective sur $\R_+^*$.

\begin{remarque}
Au lycée, on fait passer l'idée du cosinus et du sinus en disant que l'on \og enroule l'axe réel sur le cercle trigonométrique\fg, comme on enroulerait une ficelle sur une bobine. C'est une très bonne façon d'expliquer, qui remplit son objectif, mais \og enrouler\fg{} n'est pas vraiment un terme mathématique...
\end{remarque}

\begin{mdframed}
Attention : la construction rigoureuse de ces fonctions n'a a priori pas encore été faite ! Il serait en théorie possible de la faire ici, mais il faudrait plusieurs chapitres du cours d'analyse, et ce serait contre-productif d'un point de vue pédagogique.

Le fait d'admettre l'existence et les propriétés classiques de ces fonctions ne crée pas de trou logique dans la suite. Tout ce qui a été admis pourra être démontré au second semestre sans utiliser le cours sur les nombres complexes, il n'y pas de cercle vicieux.
\end{mdframed}



\begin{definition}
Soit $z \in \C$, $a = Re(z)$ et $b = Im(z)$. Alors, on définit la fonction $exp : \C\to \C$ par $exp(z) = e^a (\cos(b)+i \sin(b))$. On écrit la plupart du temps $e^z$ au lieu de $exp(z)$, pour des raisons qui apparaitront plus bas.
\end{definition}


\begin{exemples}
$e^0=1$, $e^{i\pi/2} = i$, $e^{i\pi}=-1$, $e^{3i\pi/2}=-i$ et $e^{2i\pi}=1$. 
\end{exemples}

\begin{attention}
L'exponentielle complexe est une fonction de $\C$ dans $\C$, donc ça n'a aucun sens de dire qu'elle est \og strictement croissante\fg, ou encore \og strictement positive\fg. 
\end{attention}
\begin{exo}
Montrer que l'exponentielle complexe n'est pas injective, à la différence de l'exponentielle réelle.
\end{exo}


\begin{proposition}
L'exponentielle complexe vérifie les propriétés fondamentales suivantes:
\begin{enumerate}
\item L'exponentielle complexe prolonge l'exponentielle réelle.
\item $\forall z\in\C, |e^z| = e^{Re(z)}$, en particulier l'exponentielle complexe ne s'annule jamais.
\item $\forall (z,z')\in\C^2, e^{z+z'} = e^ze^{z'}$.
\item $\forall z\in\C, e^z=1 \iff z \in 2i\pi\Z$.
\end{enumerate}
\end{proposition}
\begin{proof}
Exercice. Découle de la définition et des formules usuelles pour l'exponentielle réelle et les fonctions trigonométriques, rappelées plus haut.
\end{proof}

L'exponentielle complexe vérifie également un certain nombre d'autres propriétés importantes, que l'on peut démontrer à l'aide de la définition ou directement à l'aide de la proposition précédente, sans utiliser la définition.

\begin{proposition}
\begin{enumerate}
\item $\forall z\in\C, \forall n\in \Z, \left(e^z\right)^n = e^{nz}$ (et on n'élève \underline{jamais} un complexe à une puissance non entière!);
\item $\forall z\in\C, e^{\overline{z}} = \overline{e^z}$;
\end{enumerate}
\end{proposition}

\begin{proof}
\begin{enumerate}
\item Récurrence immédiate en utilisant les propriétés précédentes.
\item Définition, ou simple conséquence de la proposition :  
\[ e^z\overline{e^z}=\abs{e^z}^2=\left(e^{\Re(z)}\right)^2 = e^{2\Re(z)}, \]
ce qui donne $\overline{e^z}=e^{2\Re(z)-z} = e^{\overline z}$.
\end{enumerate}
\end{proof}

\begin{proposition}
Soient $z$ et $z'$ des complexes. Alors
\[e^z = e^{z'} \Leftrightarrow \left(Re(z) = Re(z') \text{ et } Im(z)\equiv Im(z')\:[2\pi]\right).\]
\end{proposition}
\begin{proof}
On a :
\begin{align*}
e^z = e^{z'} & \Leftrightarrow e^{z-z'} = 1\\
 & \Leftrightarrow z-z' \in 2i\pi\Z\\
  & \Leftrightarrow \left( Re(z-z')=0 \text{ et } Im(z-z') \in 2\pi\Z\right)\\
  & \Leftrightarrow \left( Re(z) = Re(z') \text{ et } Im(z)\equiv Im(z')\:[2\pi]\right).
\end{align*}
\end{proof}


On a vu plus haut que l'exponentielle complexe n'est pas injective. Comme elle ne s'annule jamais, elle n'est pas surjective sur $\C$. Cependant, elle l'est en corestriction à $\C^*$, c'est-à-dire que tout complexe non nul possède un antécédent par l'exponentielle complexe:

\begin{proposition}
L'image de l'exponentielle complexe est $\C^*$ :
\[ \forall w\in\C^*, \exists z\in\C, e^z=w\]
\end{proposition}
\begin{proof}
Soit $w\in\C^*$. Notons $z=|w|$. Alors $w'=\frac{w}{r}\in\U$. Il existe\footnote{C'est ici que l'on utilise une propriété forte des fonctions cosinus et sinus que l'on n'a pas démontrée entièrement, parce qu'au fond on n'a jamais défini correctement le cosinus et le sinus. On peut choisir $\theta = \arccos(\Re(w'))$ ou son opposé mais c'est pareil, la construction de l'arccosinus n'a pas été faite en détail et dépend évidemment d'une définition rigoureuse du cosinus, de sa continuité etc.} alors $\theta\in\R$ tel que $w'=\cos(\theta)+i\sin(\theta)$. On en déduit que $w'=e^{i\theta}$.

Mais alors, on peut écrire $w=re^{i\theta}$. Pour finir, comme $r>0$, c'est l'exponentielle\footnote{Pareil, on utilise des propriétés admises sur l'exponentielle réelle et le logarithme qui n'ont pas été totalement démontrées.} d'un certain réel $l$, à savoir $l = \ln(r)$. Finalement, $w=e^le^{i\theta} = e^{l+i\theta}$. En posant $z=l+i\theta$, on a bien
\[ w = e^z.\]
\end{proof}

\begin{mdframed}
Pas plus qu'il n'existe de fonction \og racine carrée complexe\fg{} raisonnable, il n'existe de fonction \og logarithme complexe\fg. En tout cas, pas au sens des applications usuelles. Pour ceux qui continueront en maths jusqu'au M1 ou M2, vous apprendrez le fin mot de l'histoire avec les \emph{surfaces de Riemann}. Dans le cours d'analyse complexe de L3, vous commencerez à vous frotter de loin à ces objets, avec l'introduction de la \og détermination principale du logarithme complexe\fg, et de quelques chapitres de cours sur les \emph{revêtements}. Patience !

Jusqu'à ce moment, en aucun cas vous ne pouvez prendre le \og logarithme\fg{} d'un nombre complexe. Les notations $\log$ ou $\ln$ ne sont valides que devant un nombre réel strictement positif, la règle est toujours la même.
\end{mdframed}




\section{Trigonométrie}
\label{sec:trigo}
Toutes les formules de trigonométrie classiques peuvent se démontrer en utilisant les propriétés de l'exponentielle complexe. Tous les symboles non définis désignent des réels. À faire en exercice.

\paragraph{Cercle}
\begin{remarque}
On rappelle que l'équation du cercle, ce n'est rien d'autre qu'une reformulation du théorème de Pythagore
\end{remarque}
Un point de coordonnées $(\cos x,\sin x)$ appartient au cercle unité :

\[ \cos^2x+\sin^2x = 1\]
\begin{remarque}
Les expressions $\sqrt{1-\cos^2 x}$ et $\sqrt{1 - \sin^2 x}$ doivent être reconnues instantanément. Attention aux pièges, voir exercice ci-dessous.
\end{remarque}
\begin{exo}
Montrer que l'assertion
\[ \forall x\in \R, \sqrt{1-\cos^2 x} = \sin x\]
est... fausse. Quelle est la formule correcte ?
\end{exo}


\paragraph{Somme}

\begin{align*}
\cos(a+b) &= \cos a \cos b - \sin a \sin b\\
\sin(a+b )&= \sin a \cos b + \cos a \sin b\\
\end{align*}

On en déduit, en appliquant ces formules lorsque $b=a$, les formules pour l'angle double:

\begin{align*}
\cos(2a) &= \cos^2 a - \sin^2 a \\
&= 2\cos^2 a-1 \\
&= 1-2\sin^2 a\\
\sin(2a)&= 2\sin a \cos a \\
\end{align*}

\begin{exo}
Trouver une formule pour $\cos(3a)$ et $\sin(3a)$.
\end{exo}

\paragraph{Symétries de translation}

Les fonctions cosinus et sinus sont $2\pi$-périodiques.
\begin{remarque}
Géométriquement, ceci signifie que leur graphe est invariant par translation de $2\pi$ selon l'axe des abscisses.
\end{remarque}


\begin{align*}
\cos(x+2\pi)&=\cos x\\
\sin(x+2\pi)&=\sin x
\end{align*}

\paragraph{Symétries axiales}

La fonction cosinus est paire: $ \cos(-x)=\cos(x)$.
\begin{remarque}
Une fonction est paire ssi son graphe admet la droite d'équation $x=0$ comme axe de symétrie.
\end{remarque}
\begin{exo}
Montrer que la fonction sinus n'est pas paire.
\end{exo}

Le graphe du sinus admet la droite (verticale) d'équation $x=\pi/2$ comme axe de symétrie :
\[ \sin(\pi-x)=\sin(x) \]

\paragraph{Symétries centrales}

La fonction sinus est impaire : $\sin(-x)=-\sin(x)$.
\begin{remarque}
Une fonction est impaire ssi son graphe admet le point de coordonnées $(0,0)$ comme centre de symétrie.
\end{remarque}
\begin{exo}
Montrer que la fonction cosinus n'est pas impaire.
\end{exo}

Le point de coordonnées $(\pi/2,0)$ est un centre de symétrie pour le graphe du cosinus, ce qui s'écrit de la façon suivante:
\[ \cos(\pi-x)=-\cos(x)\]


\begin{exo}
Déduire des formules de parité et imparité une formule pour $\cos(a-b)$ et $\sin(a-b)$. (À apprendre par c\oe ur également.)
\end{exo}

\paragraph{Symétries glissées\footnote{Une symétrie glissée est une symétrie axiale suivie d'une certaine translation}}
\begin{exo}
Montrer que les fonctions sinus et cosinus ne sont pas $\pi$-périodiques.
\end{exo}

\begin{align*}
\cos(x+\pi) &= -\cos x\\
\sin(x+\pi) &= -\sin x\\
\end{align*}

Ces formules se déduisent des précédentes.

\paragraph{Passage du cosinus au sinus}

\[ \cos(x)=\sin(\pi/2-x)\]
\[ \sin(x)=\cos(\pi/2-x)\]






\section{Racines de l'unité}
\label{sec:cyclotomie}
\subsection{Racines $n$-èmes de l'unité, avec $n\in \N^*$}

\begin{definition}
Soit $n\in\N^*$.
\begin{enumerate}
\item Si $z\in \C$, on dit que $z$ est une \emph{racine $n$-ème de l'unité} si $z^n=1$.
\item On note $\U_n$ l' ensemble des racines $n$-ème de l'unité:
\[ \U_n=\ensemble{z\in\C}{z^n=1}.\]
\end{enumerate}
\end{definition}

Autrement dit, si $z\in \C$ et $n\in\N^*$, les assertions \og $z\in\U_n$\fg{} et \og$z^n=1$\fg{} sont équivalentes, par définition de $\U_n$. 

\begin{attention}
Quand on écrit \og $\U_n$\fg{} ou que l'on parle de racines $n$-èmes de l'unité, il faut avoir déclaré le symbole $n$ auparavant.
\end{attention}


\begin{exemples}
Lorsque $n\in \N^*$ est petit, les ensembles $\U_n$ sont faciles à déterminer :
\begin{enumerate}
\item On a $\U_1=\ensemble{z\in\C}{z=1}=\{1\}$.
\item On a $\U_2=\ensemble{z\in\C}{z^2=1}=\{-1,1\}$.
\item Si $z\in\C$, alors on a la suite d'équivalences
\begin{align*}
z\in\U_3 &\iff z^3=1 \\
&\iff z^3-1=0 \\
&\iff (z-1)(z^2+z+1)=0 \\
&\iff (z-1)(z-j)(z-\bar j)=0\\
&\iff z\in \set{1,j,\bar j}
\end{align*}
On en déduit que $\U_3 = \set{1,j,\bar j}$.
\item De même, si $z\in\C$, alors on a la suite d'équivalences
\begin{align*}
z\in\U_4 &\iff z^4=1 \\
&\iff z^4-1=0 \\
&\iff (z^2-1)(z^2+1)=0 \\
&\iff (z-1)(z-1)(z-i)(z+i)=0\\
&\iff z\in \set{1,i,-1,-i}
\end{align*}
On en déduit que $\U_4 = \set{1,i,-1,-i}$.
\end{enumerate}
\end{exemples}

\begin{remarque}
Lorsque $n$ décrit $\N^*$, les différents ensembles $\U_n$ ne sont pas disjoints, ils peuvent avoir certains éléments en commun. Noter par exemple que $1$ appartient à tous les $\U_n$, que $-1$ appartient à $\U_2$ et aussi à $\U_4$ (mais pas à $\U_3$), etc.% cas général en exercice.
\end{remarque}

\begin{proposition}
Soient $n$ et $p$ dans $\in\N^*$. Si $p\in\N^*$ est un multiple de $n$, alors $\U_n\subseteq \U_p$.
\end{proposition}
\begin{proof}
Soit $z\in \U_n$. Montrons que $z\in \U_p$.
Par définition du fait d'être multiple, il existe $k\in \N^*$ tel que $p=kn$. Alors, on peut écrire
\[ z^p=z^{kn}=z^{nk}=(z^n)^k=1^k=1.\]
Donc $z\in \U_p$.
\end{proof}

(Une bonne question, que l'on peut commencer à se poser dès maintenant, est de savoir si la réciproque est vraie. Réponse plus tard.)



\subsection{Racines de l'unité}

Toutes les définitions précédentes ne concernent que les racines $n$-èmes de l'unité avec un $n$ fixé (c'est-à-dire préalablement déclaré). Il est cependant utile de  pouvoir travailler avec des nombres complexes qui sont \og racine $n$-ème de l'unité  pour un certain $n$\fg{}, mais sans avoir à préciser de tel entier $n$, du moment que l'on sait qu'il en existe bien un. Pour cela, on introduit la définition suivante :

\begin{definition}
\begin{enumerate}
\item Soit $z\in \C$. On dit que $z$ est une \emph{racine de l'unité} s'il existe $n\in \N^*$ tel que $z$ soit une racine $n$-ème de l'unité. 
\item On note\footnote{Cette notation n'est pas tout à fait standard, mais c'est celle qui est la plus compatible avec la notation $\U_n$, qui elle-même sera progressivement abandonnée au profit de $\mu_n(\C)$ à partir de l'année de L3.} $\U_\infty$ l'ensemble des racines de l'unité. Autrement dit, avec plusieurs reformulations équivalentes :
\begin{align*}
\U_\infty &= \left\{z\in \C \:\middle\vert\: \text{$z$ est racine de l'unité}\right\}\\
&= \left\{z\in \C\:\middle\vert\:\exists n\in \N^*, z^n=1\right\}\\
&= \left\{z\in \C\:\middle\vert\:\exists n\in \N^*, z\in \U_n\right\}\\
&= \bigcup_{n\in \N^*} \U_n
\end{align*}
\end{enumerate}
\end{definition}

\begin{attention}
L'ensemble $\U_\infty$ est beaucoup plus complexe que les ensembles $\U_n$, lorsque $n\in \N^*$. Par exemple, on verra plus tard que chaque $\U_n$ est un ensemble fini, alors que $\U_\infty$ est un ensemble infini. Il est conseillé d'éviter au maximum de faire appel à $\U_\infty$ et d'essayer de n'utiliser que les ensembles $\U_n$ lorsque c'est possible. (Cela dit, on démontre tout de même ici un certain nombre de propriétés élémentaires de l'ensemble $\U_\infty$.)
\end{attention}



Être racine de l'unité, c'est donc être racine $n$-ème de l'unité \og pour un certain $n$\fg. Pour démontrer qu'un nombre complexe est racine de l'unité, il suffit donc de montrer qu'il existe $n\in\N^*$ tel que $z$ est racine $n$-ème de l'unité. En général, on trouve un tel $n$ explicitement, comme dans les exemples suivants.

\begin{exemple}
\begin{enumerate}
\item Le nombre complexe $i$ est une racine de l'unité. En effet, on a $i^4=1$, donc $i$ est racine quatrième de l'unité.
\item Le nombre complexe $Z=\frac{\sqrt 3+i}{2}$ est une racine de l'unité. En effet, on a $Z=e^{i\pi/6}$, donc $Z^{12}=1$. On en déduit que $Z$ est racine douzième de l'unité.
\end{enumerate}
\end{exemple}

\begin{remarque}
Dire que $z$ est racine de l'unité c'est dire qu'il est racine de l'unité \og pour un certain $n\in\N^*$\fg, mais ce $n$ n'est pas unique : par exemple, $i$ est une racine quatrième de l'unité, mais c'est aussi une racine huitième de l'unité, pusque l'on a bien $i^8=1$. Si $Z$ est une racine de l'unité, déterminer le \og plus petit $n$\fg{} pour lequel $Z$ est racine $n$-ème est un problème intéressant sur lequel on reviendra.
\end{remarque}

\subsection{Premiers résultats de structure}

Nous avons déjà décrit les ensembles $\U_n$ pour $n\in \llbracket 1,4\rrbracket$. La détermination des ensembles $\U_n$ pour $n \geq 5$ est moins immédiate. Auparavant, on peut déjà démontrer une certaine quantité de résultats sur $\U_n$ en exploitant simplement sa définition.

\begin{proposition}
Soit $n\in\N^*$. Alors $\U_n \subseteq \U$. Autrement dit, toute racine $n$-ème de l'unité est de module un (et en particulier non nulle).
\end{proposition}
\begin{proof}
Soit $z\in\U_n$. On a donc $z^n=1$. En prenant le module de cette équation, on obtient en particulier $\abs{z^n}=\abs{1}$ c'est-à-dire $\abs{z}^n=1$. Comme $\abs{z}$ est un réel positif, ceci implique que $\abs{z}=1$.
\end{proof}

\begin{corollaire}
On a l'inclusion $\U_\infty\subseteq\U$.
\end{corollaire}
\begin{proof}
Soit $z\in \U_\infty$. Montrons que $z\in\U$. Par définition, il existe $n\in\N^*$ tel que $z\in\U_n$. Comme d'après la proposition précédente, on a $\U_n\subseteq \U$, on en déduit $z\in\U$. 
\end{proof}

\begin{proposition}[Stabilité par produit et inverse]
Soit $n\in\N^*$. Alors:
\begin{enumerate}
\item L'ensemble $\U_n$ est stable par produit, autrement dit si deux onmbres complexes sont dans $\U_n$, alors leur produit aussi. En langage mathématique : $\forall z, w\in\U_n, zw\in\U_n$.
\item L'ensemble $U_n$ est stable par inverse, autrement dit si un nombre complexe est dans $\U_n$, (il est non nul et) son inverse appartient également à $\U_n$. En langage mathématique : $\forall z\in\U_n, \frac{1}{z} \in\U_n$. (Remarquer que l'on a déjà montré qu'une racine $n$-ème de l'unité n'est jamais nulle.)
\end{enumerate}
\end{proposition}
\begin{remarque}
Dans les mois qui viennent, cette proposition sera résumée par la phrase \og Si $n\in\N^*$, alors $\U_n$ est un groupe multiplicatif.\fg
\end{remarque}

\begin{proof}
\begin{enumerate}
\item Soient $z, w\in\U_n$. On a donc $z^n=1$ et $w^n=1$. On en déduit que 
\[ (zw)^n=z^nw^n=1.\]
Donc $zw\in\U_n$.
\item Soit $z\in\U_n$. On a donc $z^n=1$. Rappelons encore une fois que ceci implique que $z$ est non nul. On peut donc former son inverse $a/z$, et cet inverse vérifie:
\[ \left(\frac{1}{z}\right)^n = \frac{1}{z^n} = \frac{1}{1} = 1.\]
Donc $\frac{1}{z} \in\U_n$.
\end{enumerate}
\end{proof}

Les trois propriétés se résument en disant que \og la multiplication munit $\U_n$ d'une structure de groupe\footnote{Il existe une définition générale de \emph{groupe} : il s'agit d'un ensemble muni d'une loi de composition, qui vérifie certaines propriétés. Les groupe des racines $n$-èmes de l'unité est un des premiers exemples de groupes que l'on croise.}\fg, ou bien simplement que \og $\U_n$ est un groupe multiplicatif\fg. (En toute rigueur la première propriété devrait être remplacée par quelque chose d'autre, mais ici ce n'est pas très important.)

Dans la suite, on écrira donc souvent \og groupe des racines $n$-èmes de l'unité\fg{} au lieu d'écrire \og ensemble des racines $n$-èmes de l'unité\fg.

Le même type de propriété est vérifié pour l'ensemble des racines de l'unité $\U_\infty$ : 

\begin{proposition}[Stabilité par produit et inverse]
Les propriétés suivantes sont vérifiées :
\begin{enumerate}
\item (Stabilité par produit) $\forall z, w\in\U_\infty, zw\in\U_\infty$;
\item (Stabilité par inverse) $\forall z\in\U_\infty, \frac{1}{z} \in\U_\infty$. (Remarquer que l'on a déjà montré qu'une racine de l'unité n'est jamais nulle.)
\end{enumerate}
\end{proposition}
\begin{remarque}
Dans les mois qui viennent, cette proposition sera résumée par la phrase \og $\U_\infty$ est un groupe multiplicatif.\fg
\end{remarque}

\begin{proof}
\begin{enumerate}
\item Soient $z, w\in\U_\infty$. Par définition, il existe $n\in\N^*$ tel que $z^n=1$, et il existe $p\in \N^*$ tel que $w^p=1$. Posons $q=np$. On a 
\[ z^q=z^{np}=(z^n)^p=1^p=1,\]
donc $z\in \U_q$. D'autre part, on a 
\[ w^q = w^{np} = w^{pn} = (w^p)^n = 1^n=1,\]
donc $w\in \U_q$. D'après la proposition précédente, $\U_q$ est stable par produit, donc $zw \in \U_q$. Et donc $zw\in\U_\infty$.
\item Ce point est plus simple que le précédent. Soit $z\in\U_\infty$. Soit $n\in\N^*$ tel que $z^n=1$. D'après la proposition précédente, $\U_n$ est stable par inverse et donc $\frac{1}{z} \in \U_n$. Donc $\frac{1}{z} \in\U_\infty$.
\end{enumerate}
\end{proof}

\begin{remarque}
Dans la preuve de la stabilité par produit, il est important de comprendre que $n$ et $p$ n'ont aucune raison d'être égaux, en général. Le point important de la preuve est de trouver un entier $q$ tel que $zw \in \U_q$, et pour cela, le plus simple est de trouver un entier $q$ tel que $z\in \U_q$ et $w\in \U_q$, puis d'utiliser la stabilité par produit. Le choix effectué ici (prendre $q=np$) est le plus simple, mais il y a plus \og efficace\fg : on aurait pu prendre $q=\ppcm(n,p)$, ce fournit un meilleur (plus petit) choix pour $q$. Prendre le produit au lieu du ppcm permet d'avoir une preuve un peu moins optimale, mais qui marche et qui évite de parler de ppcm.
\end{remarque}

\begin{remarque}
Faisons \og tourner\fg{} la preuve de stabilité par produit pour le cas particulier $z=i$ et $w=j$. On cherche à montrer que $zw$ est racine de l'unité. On sait que $i$ est racine $n$-ème de l'unité avec $n=4$. De même, on sait que $j$ est racine $p$-ème de l'unité avec $p=3$. La preuve nous fait donc poser $q=np=12$. Les nombres complexes $i$ et $j$ sont tous deux racines $12$-èmes de l'unité, donc leur produit est également racine $12$-ème de l'unité. (Dans cet exemple, on ne voit pas la différence entre produit et ppcm, car $4$ et $3$ sont premiers entre eux.)
\end{remarque}


\begin{proposition}
Soit $n\in\N^*$, et $a\in\U_n$. L'application $M_a : \begin{cases} \U_n\to \U_n,\\z\mapsto az\end{cases}$ est bien définie, et bijective.
\end{proposition}
\begin{proof}
Cela découle de la proposition précédente : le fait que l'application soit bien définie, en particulier qu'elle soit bien à valeurs dans $\U_n$, est une conséquence de la stabilité par produit. Elle est bijective car elle possède ue application réciproque, à savoir $\begin{cases} \U_n\to \U_n,\\z\mapsto \frac{z}{a}\end{cases}$.
\end{proof}

Jusqu'à présent, on a montré une certaine quantité de résultats sur $\U_n$, mais on n'a pas encore montré qu'il contenait des éléments autres que $1$. Il est raisonnable de commencer à se poser cette question, puisque ce serait quand même dommage d'écrire un chapitre entier sur l'ensemble $\{1\}$. En fait, on déterminera un peu plus tard la totalité des éléments de $\U_n$. Pour l'instant, on peut faire les remarques suivantes:
\begin{enumerate}
\item Pour $n=1$, $2$, $3$ et $4$ on a déterminé $\U_n$ et vu que c'était un ensemble fini de cardinal $1$, $2$, $3$ et $4$. On peut donc se douter qu'en général, $\U_n$ est un ensemble fini de cardinal $n$, mais ceci n'est bien sûr pas une preuve.
\item Si $n$ est pair, $(-1)^n=1$ et donc $-1 \in\U_n$, ce qui montre que dans ce cas, $\U_n$ est au minimum de cardinal $2$.
\item Une conséquence de la proposition est que si $z\in\U_n$, alors pour tout $k\in\Z$, le nombre complexe $z^k$ appartient également à $\U_n$ : en effet, $\left(z^k\right)^n=\left(z^n\right)^k=1^k=1$. Attention, ceci ne montre pas que $\U_n$ est infini, puisque ces éléments ne sont pas tous distincts : par exemple, comme $z^n=1$, on a $z=z^{n+1}$, $z^2=z^{n+2}$ etc. Cette remarque permet donc, étant donné $z\in\U_n$, de trouver potentiellement $n$ éléments (dans le meilleur des cas) dans $\U_n$ (dont $1$). Dans le meilleur des cas, car tout dépend de $z$ : par exemple si $z=1$, rajouter ses puissances successives ne donne pas grand chose. Pour $\U_4$, si on considère les puissances de $z=-1$, on n'obtient que $-1, 1, -1, 1, \cdots$ et on ne récupère jamais $i$ et $-i$. Donc prendre les puissances d'un élément n'épuise pas forcément $\U_n$ : parfois oui, parfois non, et étudier quand est-ce que c'est le cas est une question intéressante qui sera traitée plus tard.
\item On peut aussi remarquer que si $z\in\U_n$, alors $\overline z$ aussi puisque $\overline z^n=\overline{z^n}=1$. Mais cette remarque n'apporte rien de nouveau par rapport à la précédent puisque pour un nombre complexe de module un, $\overline z = \frac1z$.
\item On peut enfin remarquer que si $n\geq 2$, l'ensemble $\U_n$ contient au moins $e^{2i\pi/n}$, puisque $\left(e^{2i\pi/n}\right)^n=e^{2i\pi}=1$. Et si $n\geq 2$, $e^{2i\pi/n}$ est bien différent de $1$. On peut ensuite considérer ses puissances successives. Cette piste est fructueuse et on la poursuivra un peu plus tard. 
\end{enumerate}


\begin{proposition}
Soit $n\geq 2$ un entier. Alors la somme $S$ des racines $n$-èmes de l'unité est nulle, autrement dit:
\[ S=\sum_{z\in \U_n}z = 0.\]
\end{proposition}
\begin{proof}
Soit $a\in \U_n$, différent de $1$. On va montrer que $aS=S$, ce qui est équivalent à $(a-1)S=0$ puis à $S=0$ puisque $a\neq 1$. On procède ainsi :
\[ aS=a\sum_{z\in \U_n}z = \sum_{z\in \U_n} az = \sum_{z\in \U_n} M_a(z),\]
où $M_a$ est l'application $\begin{cases} \U_n\to \U_n,\\z\mapsto az\end{cases}$ introduite plus haut, de multiplication par $a$. On a montré qu'elle est bijective, et donc sommer les $M_a(z)$ lorsque $z$ décrit $\U_n$ revient à sommer simplement les $z$ lorsque $z$ décrit $\U_n$, mais dans un ordre différent, contrôlé par la bijection $M_a$. La somme est donc identique, c'est-à-dire $aS=S$, et donc $S=0$ comme expliqué plus haut.
\end{proof}

\begin{center}
\begin{mdframed}
Cette preuve est belle mais peut sembler difficile en première lecture (utilisation d'une bijection, de sommation indexée par un ensemble abstrait, théorème qui porte sur des éléments d'un ensemble dont on ignore encore presque tout). Dans la suite, on déterminera explicitement les éléments de $\U_n$ et on démontrera ce résultat d'une manière plus concrète. Mais il est important de lire et d'apprendre à apprécier des preuves abstraites, qui réussissent à démontrer des résultats avec très peu d'informations préalables. Les preuves abstraites sont également plus faciles à réutiliser et adapter dans d'autres contextes, ce qui fait gagner du temps.
\end{mdframed}
\end{center}

%\begin{proposition}
%Soit $n\in\N^*$ et $z\in \U_n$. Alors $\overline{z} \in\U_n$.
%\end{proposition}
%\begin{proof}
%On a : 
%\begin{align*}
%\left(\overline{z}\right)^n &= \overline{z^n} \quad \text{(multiplicativité de la congaison)}\\
%&=\overline{1}=1.
%\end{align*}
%Donc $\overline z \in\U_n$.
%\end{proof}

Enfin, avant de déterminer les éléments de $\U_n$ de manière explicite, sous forme exponentielle, on peut remarquer, si l'on connaît déjà quelques résultats sur les polynômes, que l'ensemble $\U_n$ est forcément fini, de cardinal $\leq n$. En effet,en reformulant la définition de $\U_n$, on voit que c'est par définition l'ensemble des racines complexes du polynôme $X^n-1$. Ce polynôme est de degré $n$, et un polynôme de degré $n$ possède au maximum $n$ racines distinctes.

Par ailleurs, nous avons déjà remarqué que l'élément $e^{2ik\pi/n}$ appartient à $\U_n$, puisque $\left(e^{2i\pi/n}\right)^n=e^{2i\pi}=1$. Un raisonnement semblable montre que pour tout $k\in\Z$, le nombre complexe $e^{2ik\pi/n}$ appartient à $\U_n$. On peut se convaincre assez rapidement que ceci nous fournit $n$ éléments disctincts de $\U_n$. En utilisant l'argument sur le nombre de racines d'un polynôme, on sait qu'il ne peut y en avoir plus et donc il n'y a pas d'autres éléments dans $\U_n$.

Cependant, comme le cours sur les polynômes n'est pas supposé connu à ce stade, nous donnerons par la suite une autre démonstration de ces résultats, plus concrète.

\subsection{Forme exponentielle des racines $n$-èmes}


\begin{proposition}
Soit $n\in\N^*$. Alors on a
\[ \U_n = \ensemble{e^{2ik\pi/n}}{k\in \llbracket 0,n-1\rrbracket}\]
\end{proposition}
\begin{proof}
Soit $z\in\C^*$, et écrivons $z=re^{i\theta}$, avec $r\in\R_+^*$ son module et $\theta\in\R$ un de ses arguments. On a la suite d'équivalences
\begin{align*}
z^n=1 &\iff r^ne^{in\theta}=1\\
&\iff \begin{cases}r^n=1\\n\theta \equiv 0\mod 2\pi\end{cases}
&\iff \begin{cases}r=1\\\theta \equiv 0\mod \frac{2\pi}{n}\end{cases}
\end{align*}
Par définition de ce qu'est une congruence, on obtient donc :

\[ \U_n = \ensemble{e^{2ik\pi/n}}{k\in \Z}\]
\end{proof}
%\begin{remarque}
%Ceci signifie que si $z\in\C$, alors \og$z\in\U_n$\fg{} est équivalente à \og$\exists k\in \llbracket 0,n-1\rrbracket, z=e^{2ik\pi/n}$.\fg
%\end{remarque}

Cette écriture explicite permet de donner une nouvelle démonstration du résultat sur la somme des racines $n$-èmes.

\begin{proposition}
Soit $n$ un entier supérieur ou égal à deux. Alors la somme $S$ des racines $n$-èmes de l'unité est nulle, autrement dit:
\[ S=\sum_{z\in \U_n}z = 0.\]
Une autre façon, plus concrète d'écrire cette formule est:
\[ S=\sum_{k=0}^{n-1}e^{2ik\pi/n} = 0\]
\end{proposition}
\begin{proof}
On reconnaît une somme géométrique de raison $e^{2i\pi/n}$ qui est différente de $1$ puisque $n\geq 2$. Cette somme vaut donc
\[ \frac{1-\left(e^{2i\pi/n}\right)^n}{1-e^{2i\pi/n}} = 0.\]
\end{proof}

\subsection{Interprétation géométrique}

\begin{proposition}Soit $n\geq 2$ un entier. 
Les éléments de $\U_n$ sont les affixes d'un polygone régulier à $n$ côtés, inscrit dans le cercle unité du plan.
\end{proposition}

(Un $2$-gone est juste un segment. Un $3$-gone régulier est un triangle équilateral, un $4$-gone régulier est un carré, etc.)


\subsection{Compléments : générateurs et racines primitives}

Dans toute cette section, $n$ désigne un entier naturel non nul.

\begin{definition}
Soit $a\in\C$. On dit que $a$ est un \emph{générateur} de $\U_n$ si
\[ \U_n = \ensemble{a^n}{n\in\Z}\]
\end{definition}

Autrement dit, si $\U_n$ est égal à l'ensemble $\set{\cdots , a^{-2}, a^{-1}, a^0=1, a, a^2, a^3, \cdots}$. En particulier, $a$ doit forcément appartenir à $\U_n$ pour en être un générateur. Remarquer aussi que $1$ n'est jamais un générateur de $\U_n$, sauf si $n=1$.

\begin{proposition}
Le nombre complexe $e^{2i\pi/n}$ est un générateur de $\U_n$.
\end{proposition}
\begin{proof}
Dire que $e^{2i\pi/n}$ est un générateur de $\U_n$, c'est dire que $\U_n=\ensemble{e^{2ik\pi/n}}{k\in \Z}$ et c'est ce que l'on a montré plus haut.
\end{proof}

\begin{exemples}
$-1$ est un générateur de $\U_2$, $j$ est un générateur de $\U_3$, $i$ est un générateur de $\U_4$, $e^{2i\pi/5}$ est un générateur de $\U_5$ etc.
\end{exemples}

Évidemment, il peut tout-à-fait y avoir plus d'un générateur : 

\begin{exercice}
Montrer que les générateurs de $\U_3$ sont $j$ et $j^2$, et que les générateurs de $\U_4$ sont $i$ et $-1$ (mais pas $-1$). Combien y a-t-il de générateurs dans $\U_6$ ? Et dans $\U_{12}$ ? Comment généraliser ces résultats ?
\end{exercice}

\begin{definition}
Les générateurs de $\U_n$ sont également appelés les racines $n$-èmes \emph{primitives} de l'unité.
\end{definition}




\chapter{Nombres complexes et géométrie}
\minitoc
\hyperlink{toc}{\retourTOC}

\section{Rappels : repères, bases, vecteurs, points (en cours)}
\label{sec:rappels_geom}
Notation pour le plan euclidien : $\mathcal P$.

Vecteurs du plan.

Notation pour l'ensemble de tous les vecteurs du plan : $\overrightarrow{\mathcal P}$.

Règles de calcul avec points et vecteurs ($\overrightarrow{\mathcal P}$ est un ev, sans le dire).

Pas de barycentres : voir L2.

Pas d'action de $\overrightarrow{\mathcal P}$ sur $\mathcal P$ ? Si, mais sans le vocabulaire des actions ?
Tout ceci sera repris dans le cours de géométrie affine de L2.

\begin{definition}
Une base de $\overrightarrow{\mathcal P}$ est un couple de vecteurs non nuls, et non colinéaires.
\end{definition}

Repères du plan. Repères orthonormés, bases orthonormées.

Tous les rappels : équations de droites, paramétrages, équations de demi-plans avec le produit scalaire, déterminant

\section{Produit scalaire (en cours)}
\label{sec:produit_scalaire}
\begin{definition}
En coordonnées dans une base qui sera alors orthonormée par définition
\end{definition}

\begin{proposition}
Changement de repère orthonormé.
\end{proposition}

\begin{proposition}
Écriture en coordonnée complexe.
\end{proposition}

\begin{proposition}
Bilinéarité, symétrie, défini positif.
\end{proposition}

\begin{definition}
Norme d'un vecteur.
\end{definition}

On peut donc définir la norme d'un vecteur grâce au produit scalaire. Mais l'inverse est également vrai  : on peut entièrement caractériser le produit scalaire uniquement grâce aux normes de vecteurs : 

\begin{proposition}
[Identités de polarisation]

\end{proposition}

\begin{definition}
Orthogonalité de deux vecteurs.
\end{definition}

\begin{proposition}
Cauchy-Schwarz
\end{proposition}

\begin{proposition}
Inégalité triangulaire
\end{proposition}



\section{Déterminant et aire orientée (en cours)}
\label{sec:determinant}
\begin{definition}
Déterminant de deux vecteurs, relativement à une base.
\end{definition}

\begin{proposition}
[Écriture en coordonnée complexe]
Soient $\vec u$ et $\vec v$ deux vecteurs du plan, d'affixes $z$ et $z'$ relativement à une base $\mathcal B$. Alors, leur déterminant (relativement à cette base) est
\[ \det(\vec u, \vec v) = \Im(\overline z\cdot z')\]
\end{proposition}
\begin{proof}
Soient $(x,y)=\Coord \vec u$ et $(x',y')=\Coord \vec v$. Alors
\[ \Im(\overline z\cdot z') = \Im\left(xx'+yy'+i(xy'-yx')\right) = xy'-yx' = \det(\vec u, \vec v).\]
\end{proof}

\begin{definition}
On dit que deux bases ont la même orientation si le déterminant de l'une dans l'autre est positif. SInon on dit qu'elles ont une orientation inverse.
\end{definition}

\begin{exemples}Soit $\mathcal B = (u,v)$ la base canonique de $\R^2$.
\begin{enumerate}
\item La base $\mathcal B'=(\vec u+\vec v, -\vec u+2\vec v)$ a la même orientation que $\mathcal B$.
\item La base $\mathcal B''=(\vec v, \vec u)$ a l'orientation inverse de celle de $\mathcal B$.
\end{enumerate}
\end{exemples}

Orientation, bases directes, indirectes

\begin{proposition}
La formule pour l'écriture du déterminant est la même dans toute base orthonormée directe.
\end{proposition}

\begin{proposition}
\begin{enumerate}
\item (bilinéarité)
\item (Le déterminant est une forme bilinéaire alternée)
\item (Le déterminant est une forme bilinéaire antisymétrique)
\end{enumerate}
\end{proposition}

Interprétation géométrique : condition de colinéarité.

Interprétation géométrique : $\lvert \det(\vec u, \vec v)\rvert $ est l'aire du parallélogramme porté par $\vec u$ et $\vec v$. (Sans les valeurs absolues, on obtient l'aire algébrique, qui tient compte de l'orientation, et vérifie de meilleures formules.)

\begin{exemple}
Soit $ABC$ le triangle dont les sommets ont pour affixe $a=1+i$, $b=5+2i$ et $c=3+4i$. Alors il a une aire égale à 
\[\mathcal A(ABC)= \frac12\left|\Im \overline{(b-a)}(c-a)\right| = \frac12\left|\Im (4-i)(2+3i)\right|=5\]
\end{exemple}



\section{Angles géométriques, angles orientés (en cours)}
\label{sec:angles}
\subsection{Angles géométriques (entre deux demi-droites)}

Cauchy-Schwarz permet de définir l'angle géométrique de deux vecteurs non nuls $\vec u$ et $\vec v$ comme

le réel $\arccos\left(\frac{\vec u\cdot \vec v}{\|\vec u\|\cdot \|\vec v\|}\right)$, puisque la parenthèse est d'après CS un réel entre $-1$ et $1$, donc dans le domaine de définition d'arccos.

Cette notion donne un objet de nature simple (un réel), qui plus est facile à calculer : 

\begin{exemples}
\end{exemples}

Par contre, la notion de comporte mal, elle ne vérifie pas de formules pratiques, elle ne permet pas de démontrer des théorèmes facilement...



\subsection{Angles orientés}

Notation
Cahier des charges : 
\begin{enumerate}
\item remplacement par vecteurs unitaires.
\item Chasles.
\item $\widehat{(\overrightarrow u, \overrightarrow u)} \equiv 0 \mod 2\pi$.
\item $\widehat{(\overrightarrow u, \overrightarrow v)} \equiv -\widehat{(\overrightarrow v, \overrightarrow u)} \mod 2\pi$.
\end{enumerate}

\begin{definition}
définition comme classe de congruence de réels modulo $2\pi$ : la classe modulo $2\pi$ des arguments de $\overline z\cdot w$, qui est aussi celle des arguments de $\frac{w}{z}$.
\end{definition}


\begin{proposition}
Le cahier des charges est respecté, autrement dit : 
\begin{enumerate}
\item Chasles.
\item $\widehat{(\overrightarrow u, \overrightarrow u)} \equiv 0 \mod 2\pi$.
\item $\widehat{(\overrightarrow u, \overrightarrow v)} \equiv -\widehat{(\overrightarrow v, \overrightarrow u)} \mod 2\pi$.
\end{enumerate}
\end{proposition}

\begin{remarque}
Remarquer la ressemblance avec les propriétés du déterminant.
\end{remarque}

\chapter{Isométries}
\minitoc
\hyperlink{toc}{\retourTOC}




\section{Généralités sur les isométries}

\begin{definition}
Soit $f : \mathcal P \to \mathcal P$. On dit que $f$ est une \emph{isométrie} si
\[ \forall M, M' \in \mathcal P, \dist(M,M') = \dist(f(M),f(M'))\]
\end{definition}

Cela signifie que si deux points sont à une certaine distance $d$ l'un de l'autre, leurs images seront exactement à la même distance $d$ l'une de l'autre.



\begin{proposition}
Une isométrie  est injective.
\end{proposition}
\begin{proof}
Soit $f$ une isométrie et soient $M$ et $M'$ des points tels que $f(M)=f(M')$. En reformulant, on a donc $\dist(f(M), f(M'))=0$. Comme $f$ est une isométrie, on en déduit $\dist(M,M')=0$, autrement dit $M=M'$. Ceci montre que $f$ est injective.
\end{proof}




\begin{proposition}
Une isométrie plane est bijective.
\end{proposition}
\begin{proof}
On admet ce résultat, dont la démonstration n'est pas immédiate (mais peut néanmoins se faire de manière relativement élémentaire). Lorsque ce sera possible, on montrera à la main que telle ou telle isométrie est bien surjective pour limiter au maximum la part de résultats admis.
\end{proof}

\begin{proposition}
\begin{enumerate}
\item La composée de deux isométries est une isométrie. 
\item L'application réciproque d'une isométrie est une isométrie.
\end{enumerate}
On résume ces deux propriétés en disant que les \og isométries forment un \emph{groupe} pour la composition\fg.
\end{proposition}
\begin{proof}
\begin{enumerate}
\item Soient $f$ et $g$ des isométries. Montrons que $g\circ f$ est également une isométrie.
Soient $A$ et $B$ deux points du plan. On a alors:
\begin{align*}
\dist(A,B)
&= \dist(f(A),f(B) & \text{car $f$ est une isométrie}\\
&= \dist(g(f(A)),g(f(B)) & \text{car $g$ est une isométrie}\\
\end{align*}
D'où $\dist(A,B) = \dist(g\circ f(A),g\circ f(B))$, ce qui montre que $g\circ f$ est une isométrie.
\item Soit $f$ une isométrie et $g$ son application réciproque (on a admis qu'une isométrie était bijective, elle possède donc une bijection réciproque). Soient $A$ et $B$ deux points. Alors  on a 
\begin{align*}
\dist(g(A),g(B))
&= \dist(f(g(A)),f(g(B))) & \text{car $f$ est une isométrie}\\
&= \dist(A,B)& \text{car $f\circ g=\Id_{\mathcal P}$}\\
\end{align*}
Ceci montre que $g$ est une isométrie.
\end{enumerate}
\end{proof}

On pourrait dire bien plus de choses sur les isométries de façon générale, mais cela augmenterait sensiblement la technicité du texte. Le parti pris dans ce cours est d'aller en priorité vers l'étude d'exemples concrets (rotations, translations) puis vers les similitudes directes, et de reporter à plus tard (L2) l'étude générale des isométries planes (directes ou indirectes) et  leur classification, une fois que les techniques d'algèbre linéaire (et bilinéaire) seront à disposition.

On ajoute toutefois quelques petits résultats généraux sur les isométries, que l'on peut démontrer simplement avec des outils élémentaires.

\begin{proposition}
Une isométrie conserve les milieux. Autrement dit, si $f : \mathcal P\to \mathcal P$ et $A$ et $B$ sont deux points distincts et $M$ le milieu de $[AB]$, alors $f(M)$ est le milieu de $[f(A)f(B)]$.
\end{proposition}
\begin{proof}
On a $AM=MB=\frac12AB$ et comme $f$ est une isométrie, on a également $f(A)f(M)=f(M)f(B)=\frac12f(A)f(B)$. D'après l'inégalité triangulaire, ceci montre que $f(M)$ est le milieu de $[f(A)f(B)]$.
\end{proof}
%\begin{remarque}
%Pour ceux qui savent ce qu'est un barycentre, les isométries conservent également les barycentres, pas simplement les milieux.
%\end{remarque}

\begin{proposition}[Conservation de l'alignement]
Soit $f : \mathcal P\to \mathcal P$ une isométrie, $A$, $B$ et $C$ trois points distincts et $A'$, $B'$ et $C'$ leurs images par $f$. Alors $A$, $B$ et $C$ sont alignés si et seulement si $A'$, $B'$ et $C'$ sont alignés. Si c'est le cas, l'ordre d'alignement est le même.
\end{proposition}
\begin{proof}
Si $A$, $B$ et $C$ sont alignés, il y en a un des trois qui est entre les deux autres. Mettons que ce soit $B$, et que l'on ait donc  $AC=AB+BC$. On en déduit que $A'C'=A'B'+B'C'$ et par le cas d'égalité de l'inégalité triangulaire, on en déduit que $A'$, $B'$ et $C'$ sont alignés, avec $B'$ entre $A'$ et $C'$.

Réciproquement, si $A$, $B$ et $C$ ne sont pas alignés, le triangle $ABC$ n'est pas plat et on a trois  inégalités strictes $AB<AC+CB$, $BC<BA+AC$ et $CA<CB+BA$. On en déduit les trois inégalités strictes $A'B'<A'C'+C'B'$, $B'C'<B'A'+A'C'$ et $C'A'<C'B'+B'A'$. Ceci montre que $A'$, $B'$ et $C'$ ne sont pas alignés. 
\end{proof}

\begin{proposition}[Conservation des angles droits]
Soit $f : \mathcal P\to \mathcal P$ une isométrie, $A$, $B$ et $C$ trois points distincts et $A'$, $B'$ et $C'$ leurs images par $f$. Si $ABC$ est rectangle en $A$, alors $A'B'C'$ est rectangle en $A'$.
\end{proposition}
\begin{proof}
Une preuve élémentaire utilise le théorème de Pythagore, sans utiliser le langage du produit scalaire. Si $ABC$ est rectangle en $A$, on a par Pythagore $BC^2=AB^2+AC^2$. Comme $f$ est une isométrie, on en déduit $B'C'^2=A'B'^2+A'C'^2$ et par la réciproque de Pythagore, $A'B'C'$ est rectangle en $A'$.
\end{proof}

\begin{proposition}
L'image d'une droite par une isométrie est une droite.
\end{proposition}
\begin{proof}
Exercice : on a déjà montré que l'image d'une droite est incluse dans une droite, par la propriété d'alignement. Il reste à justifier que l'image est la droite toute entière. C'est laissé en exercice.
\end{proof}

\begin{proposition}
Soit $f : \mathcal P\to \mathcal P$ une isométrie.
\begin{enumerate}
\item Si deux droites sont parallèles et distinctes\footnote{Par convention, une droite est toujours parallèle à elle-même, pour que le parallélisme soit une \emph{relation d'équivalence}.}, leurs images par $f$ sont deux droites parallèles et distinctes.
\item Si deux droites sont orthogonales, leurs images par $f$ sont deux droites orthogonales.
\end{enumerate}
\end{proposition}
\begin{proof}
\begin{enumerate}
\item Soient $\mathcal D$ et $\mathcal D'$ deux droites distinctes parallèles. Elles sont donc disjointes. Comme l'isométrie $f$ est injective, les images de deux parties disjointes quelconques sont disjointes. Par la proposition précédente, ces images sont dans le cas présent deux droites, qui sont donc disjointes et donc parallèles (et distinctes).
\item On applique les propositions précédentes : les images des deux droites sont deux droites, et comme l'image d'un triangle rectangle est rectangle on en déduit le résultat.
\end{enumerate}
\end{proof}

\begin{mdframed}
Les résultats généraux que l'on passe ici sous silence sont  : la conservation du produit scalaire et donc des angles géométriques, le caractère affine, la notion d'isométrie directe et indirecte, l'étude des points fixes, et la classification des isométries. Rendez-vous en L2 !
\end{mdframed}










\section{Quelques isométries classiques : translations, rotations, symétries}

\begin{definition}
Soit $f : \mathcal P\to \mathcal P$. On dit que $f$ est une translation s'il existe $\overrightarrow{u} \in \overrightarrow{\mathcal P}$ tel que:
\[ \forall M \in\mathcal P, \overrightarrow{Mf(M)} = \overrightarrow u\]
\end{definition}

\begin{remarque}
Si $\overrightarrow u=\overrightarrow 0$, l'application est l'identité. \textbf{L'identité est un cas particulier de translation.}
\end{remarque}


\begin{definition}
Soit $f : \mathcal P\to \mathcal P$. On dit que $f$ est une rotation s'il existe $\Omega\in\mathcal P$ et $\theta\in\R$ tels que:
\[ 
\begin{cases}
\forall M\in\mathcal P, &\Omega M = \Omega f(M)\\
\forall M\in\mathcal P \setminus\{\Omega\},& (\widehat{\overrightarrow{\Omega M},\overrightarrow{\Omega f(M)}})\equiv \theta \mod 2\pi
\end{cases}
\]
\end{definition}

%On admet que le point $\Omega$ est unique sauf si $\theta\equiv 0 \mod 2\pi$, et on l'appelle le centre de la rotation. Le réel $\theta$ n'est pas unique, mais il est unique modulo $2\pi$, on l'appelle (lui ou tout représentant de sa classe de congruence) l'angle de la rotation.

\begin{remarque}
Si $\theta\equiv 0 \mod 2\pi$, l'application est l'identité. \textbf{L'identité est un cas particulier de rotation.}
\end{remarque}

\begin{definition}
Soit $f : \mathcal P\to \mathcal P$. On dit que $f$ est une symétrie centrale s'il existe $\Omega\in\mathcal P$ tel que
\[ \forall M\in\mathcal P, \overrightarrow{\Omega f(M)}=-\overrightarrow{\Omega M}\]
\end{definition}

\begin{remarque}
Une symétrie centrale n'est rien d'autre qu'une rotation d'angle $\pi$.
\end{remarque}

\begin{definition}
Soit $f : \mathcal P\to \mathcal P$. On dit que $f$ est une symétrie axiale s'il existe une droite $\Delta$ telle que
\[ \forall M\in\mathcal P, \Delta \text{ est la médiatrice du segment } [Mf(M)]\]
\end{definition}

\begin{remarque}
L'identité est à la fois une translation, une rotation et une symétrie centrale. On dit que c'est une translation triviale, une rotation triviale, ou une symétrie triviale.
\end{remarque}


%\begin{definition}
%Soit $f : \mathcal P\to \mathcal P$. On dit que $f$ est une symétrie glissée si c'est la composée d'une symétrie axiale d'axe $\Delta$, suivie d'une translation d'un vecteur qui dirige $\Delta$.
%\end{definition}

L'unicité des points, vecteurs, angles, droites qui apparaissent dans ces définitions n'est pas claire : par exemple, rien n'indique dans la définition d'une translation sur le vecteur $\vec u$ est unique ou pas, si une rotation peut avoir plusieurs centres etc. Les définitions demandent juste l'existence de certains objets.

Dans la majorité des cas, ces paramètres qui apparaissent dans les définitions sont uniques lorsqu'ils existent, mais pas toujours. Par exemple, le réel $\theta$ qui apparaît dans la définition d'une rotation n'est jamais unique, il n'est unique que modulo $2\pi$. Plus délicat : le centre n'est pas unique car il y a une exception : l'identité, qui est une rotation d'angle nul par rapport à n'importe quel \og centre\fg. 

Pour étudier plus simplement la question de l'unicité de ces \og éléments caractéristiques\fg, il est pratique de passer en coordonnée complexe, c'est ce qui est fait dans le paragraphe suivant.

\section{Écriture des isométries directes usuelles en coordonnées complexe}

Le plan euclidien $\mathcal P$ est muni d'un repère orthonormé fixe d'une fois pour toutes. Les coordonnées cartésiennes, et les affixes, sont pris relativement à ce repère.


Si $f  : \mathcal P\to \mathcal P$ est une application, on peut, grâce au repère fixé,  associer à $f$ une unique fonction
\[ \tilde f : \C\to \C\]
qui correspond à $f$ : c'est l'écriture de $f$ en coordonnée complexe. Cette fonction vérifie la relation
\[ \Aff  f(P) = \tilde f(\Aff P)\]
Autrement dit, l'affixe de $f(M)$ est obtenu en prenant l'affixe de M, puis en appliquant la fonction $\tilde f$.

%À vrai dire, $\Aff(f)$ serait une meilleure notation que $\tilde f$ mais elle alourdirait un peu le texte.
Dans plusieurs ouvrages on note simplement $f$ au lieu de $\tilde f$ : c'est un peu gênant car le symbole $f$ désigne alors deux choses : l'application $\mathcal P\to \mathcal P$ et aussi l'application associée $\C\to\C$, mais en pratique cela reste souvent compréhensible. Dans ce cours, on essaiera dans la mesure du possible de séparer le rôle de $f$ et de $\tilde f$ et il est demandé d'en faire autant lors des évaluations.



Commençons par transposer au cadre des applications complexes les définitions données plus haut.


\begin{proposition}
Soit $f : \mathcal P\to \mathcal P$. 
\begin{enumerate}
\item C'est une isométrie si et seulement si :
\[ \forall (z,w)\in\C^2, \abs{z-w} = \abs{\tilde f(z)-\tilde f(w)}.\]
\item C'est une translation si et seulement si :
\[ \exists b\in\C, \forall z\in\C, \tilde f(z)-z=b.\]
\item C'est une rotation si et seulement si :
\[ \exists \omega \in\C, \exists \theta\in\R, \forall z\in\C, \tilde f(z)-\omega = e^{i\theta}(z-\omega).\]
\item C'est une symétrie centrale si et seulement si :
\[ \exists \omega\in\C, \forall z\in\C, \tilde f(z)-\omega=\omega-z.\]
\end{enumerate}
\end{proposition}
\begin{proof}
C'est une simple traduction des définitions, en utilisant l'interprétation géométrique du module et de l'argument.
\end{proof}

\begin{attention}
Attention à l'ordre des quantificateurs, qui est crucial.
\end{attention}




%On donnera plus bas les écritures en coordonnée complexe des symétries axiales et des symétries glissées.

\section{Unicité des éléments caractéristiques des translations et rotations}

\begin{proposition}
Soit $\phi : \C\to \C$ une translation. Il existe un unique complexe $b$ tel que 
\[ \forall z\in \C, \phi(z)=z+b\]
On l'appelle (l'affixe du) vecteur de translation de $\phi$.
\end{proposition}
\begin{proof}
Soient $b$ et $b'$ deux complexes tels que $\forall z\in\C, z+b=z+b'$. En prenant $z=0$ on obtient $b=b'$.
\end{proof}

\begin{proposition}
Soit $\rho : \C\to\C$ une rotation. Si $\rho$ n'est pas l'identité, il existe  un unique $\omega\in\C$ et un réel $\theta$ unique modulo $2\pi$, tels que 
\[ \forall z\in\C, \rho(z)=e^{i\theta}(z-\omega)+\omega.\]
On les appelle le centre et l'angle de la rotation.

Si $\rho$ est l'identité, le réel $\theta$ est également unique modulo $2\pi$ : on a alors $\theta\equiv 0 \mod 2\pi$. Par contre, dans ce cas, le point $\omega$ n'est plus unique : tout $\omega\in\C$ convient. On dit que l'identité est une rotation d'angle nul, par contre elle n'a pas de centre (ou plutôt tout point peut être considéré comme son centre).
\end{proposition}
\begin{proof}
Soient $\theta$, $\theta'$, $\omega$ et $\omega'$ tels que 

\[ \forall z\in\C, e^{i\theta}(z-\omega)+\omega = e^{i\theta'}(z-\omega')+\omega'.\]
En prenant $z=\omega$, on obtient $\omega-\omega'=e^{i\theta'}(\omega-\omega')$, c'est-à-dire $(\omega-\omega')(e^{i\theta'}-1)=0$. Ceci est équivalent à $\theta'\equiv 0\mod 2\pi$ ou $\omega=\omega'$.
\begin{enumerate}
\item Si $\theta'\equiv 0\mod 2\pi$, alors $\phi$ est l'identité. On a alors également $\theta\equiv 0 \mod 2\pi$, et tout complexe $\omega$ satisfait la définition.
\item Sinon, alors $\omega=\omega'$, dans ce cas on obtient $e^{i\theta}=e^{i\theta'}$ c'est-à-dire $\theta\equiv \theta'\mod 2\pi$.
\end{enumerate}
\end{proof}

\begin{definition}
Les éléments caractéristiques d'une rotation ou d'une translation sont :
\begin{enumerate}
\item Pour la translation, son vecteur.
\item Pour la rotation, son centre et son angle si elle est non triviale, son angle (nul) si elle est triviale.
\end{enumerate}
\end{definition}


\section{Points fixes des isométries classiques}

On rappelle (ou pas) la définition très générale suivante.

\begin{definition}[Point fixe]
Soit $E$ un ensemble, et $f : E\to E$ une application de $E$ dans lui-même. Soit $x\in E$. On dit que $x$ est un point fixe de $f$, ou que $x$ est fixe sous $f$, si $f(x)=x$.
\end{definition}

\begin{exemples}
\begin{enumerate}
\item L'identité d'un ensemble $E$, notée $\Id_E$, est par définition l'application de $E$ dans $E$ qui fixe tous les points.
\item Les points fixes de l'application $f : \R\to \R, x\mapsto 3x+5$ sont les réels $x$ vérifiant $f(x)=x$, autrement dit $3x+5=x$. Il y en a uniquement un, à savoir $x=-5/2$.
\item Une application n'a pas forcément de point fixe : l'application $g : \R\to \R, x\mapsto x+3$ n'a aucun point fixe.
\item L'application $h : \R\to \R, x\mapsto x^2$ possède deux points fixes $0$ et $1$.
\end{enumerate}
\end{exemples}
Ces exemples traitent le cas d'applications de $\R$ dans $\R$ mais ici, nous sommes plutôt intéressés par des applications de $\C$ dans $\C$, ou, de façon presque équivalente, de $\mathcal P$ dans $\mathcal P$.

\begin{proposition}
Soit $f$ une translation. Si elle n'est pas triviale (c'est-à-dire si elle n'est pas l'identité), elle n'admet aucun point fixe.
\end{proposition}
\begin{proof}
Soit $b\in\C$ tel que la translation soit représentée par $\tilde f : z\mapsto z+b$. Soit $\alpha\in\C$ un point fixe. Alors par définition $\alpha=\alpha+b$ et donc $b=0$, et donc $f$ est l'identité.
\end{proof}

\begin{proposition}
Soit $f$ une rotation. Si elle n'est pas triviale (autrement dit si elle n'est pas l'identité), alors elle admet un unique point fixe, son centre.
\end{proposition}

\begin{proof}
Soit $\omega\in\C$ et $\theta\in\R$ tels que la rotation soit représentée par $\tilde f : z\mapsto e^{i\theta}(z-\omega)+\omega$. Si la rotation est non triviale, alors $e^{i\theta}\neq 1$. Soit $\alpha\in\C$ Alors:
\begin{align*}
f(\alpha)=\alpha
&\iff e^{i\theta}(\alpha-\omega)+\omega=\alpha\\
&\iff \alpha(e^{i\theta}-1)=e^{i\theta}\omega-\omega\\
&\iff \alpha = \omega.
\end{align*}
(La division par $e^{i\theta}-1$ est licite d'après la remarque plus haut.)
\end{proof}

\begin{exercice}
On admet que la transformation $f:\mathcal P\to \mathcal P$ représentée par $\tilde f :\C\to\C,  z\mapsto iz+2+4i$ est une rotation. Montrer que son centre a pour affixe $\omega=-1+3i$.
\end{exercice}
\begin{red}
Il suffit de vérifier que $\tilde f(\omega)=\omega$.
\end{red}

\begin{mdframed}
La proposition qui suit est très utile : elle permet de montrer facilement qu'une transformation est une rotation sans avoir à en exhiber le centre.
\end{mdframed}

\begin{proposition}
Soit $a\in\C$ de module un et différent de $1$ (c'est-à-dire $a\in\U\setminus \{1\}$) et $b\in\C$. L'application $f : \mathcal P\to \mathcal P$ représentée par $\tilde f :\C\to \C,  z\mapsto az+b$ est une rotation d'angle $\arg(a)$.
\end{proposition}
\begin{proof}
On constate qu'une telle application a un unique point fixe : 
\[ z=\tilde f(z) \iff z=az+b \iff z=\frac{b}{1-a}\]
Notons $\omega=\frac{b}{1-a}$ et montrons que $f$ est une rotation dont le centre est d'affixe $\omega$ et dont l'angle est un argument de $a$. On a, pour tout $z$:
\[ a(z-\omega)+\omega=a\left(z-\frac{b}{1-a}\right)+\frac{b}{1-a}=az-\frac{ab}{1-a}+\frac{b}{1-a} = az+b = \tilde f(z).\]
\end{proof}




\begin{exercice}
La transformation $f:\mathcal P\to \mathcal P$ représentée par $\tilde f :\C\to\C,  z\mapsto -iz+3+i$ est une rotation. Déterminer son centre.
\end{exercice}

\section{Composition des isométries classiques}

\begin{proposition}[Composition et réciproques des translations]
\begin{enumerate}
\item Si $f$ et $g$ sont des translations, les deux composées $f\circ g$ et $g\circ f$ coïncident et cette application est une translation.
\item Une translation admet une application réciproque et sa réciproque est une translation.
\end{enumerate}
On résume ces propriétés en disant que les translations forment un groupe pour la composition, et que ce groupe est \emph{commutatif} (on dit aussi : \emph{abélien}).
\end{proposition}
\begin{proof}
On peut donner des preuves géométriques de cette propriété, mais utilisons les complexes. 
\begin{enumerate}
\item Soient $f$ et $g$ deux translations. Alors il existe des complexes $b$ et $b'$ tels que
\[ \forall z\in\C, \tilde f (z) = z+b \text{ et } \forall z\in\C, \tilde g(z)=z+b'\]
La composée $f \circ g$ est représentée par :
\[ \tilde f \circ \tilde g  : \begin{cases}
\C\to\C\\
z\mapsto \tilde f(\tilde g(z)) = \tilde f(z+b') = (z+b')+b = z+(b'+b)
\end{cases}\]
Un calcul semblable montre que la composée $g\circ f$ est représentée par  $z\mapsto z+b+b'$, ce qui montre que $f\circ g=g\circ f$. Cette composée est la translation dirigée par le vecteur d'affixe $b'+b$. 
\item Du point précédent on déduit que $f\circ g=\Id$ si et seulement si $b'=-b$. Ceci montre que toute translation admet une réciproque, à savoir la translation de vecteur opposé.
\end{enumerate}
\end{proof}

\begin{proposition}[Composition et réciproque des rotations de même centre]
Soit $\Omega\in\mathcal P$. \textbf{(Attention, ce point est fixé d'une fois pour toutes dans toute cette proposition.)}
\begin{enumerate}
\item Soient $f$ et $g$ deux rotations de centre $\Omega$. Alors la composée $f\circ g$ est égale à $g \circ f$. C'est une rotation de centre $\Omega$, et son angle est la somme des angles de $f$ et de $g$ (modulo $2\pi$).
\item Soit $f$ une rotation de centre $\Omega$. Alors $f$ admet une application réciproque, qui est aussi une rotation de centre $\Omega$, et d'angle opposé (modulo $2\pi$).
\end{enumerate}
\end{proposition}
\begin{proof}
\begin{enumerate}
\item Écrivons les applications en coordonnée complexe :
\[
\tilde f : z\mapsto e^{i\theta}(z-\omega)+\omega,\quad
\tilde g : z\mapsto e^{i\theta'}(z-\omega)+\omega
\]
Alors on peut calculer les deux composées:
\begin{align*}
\tilde f\circ \tilde g(z)
&=e^{i\theta}(e^{i\theta'}(z-\omega)+\omega-\omega)+\omega\\
&=e^{i(\theta+\theta')}(z-\omega)+\omega
\end{align*}
\item Du point précédent, on voit que $f\circ g=\Id_{\mathcal P}$ si et seulement si $e^{i(\theta+\theta')}=1$, autrement dit si et seulement si $\theta'\equiv -\theta\mod 2\pi$.
\end{enumerate}
\end{proof}



\begin{attention}
La composée de deux rotations n'ayant \textbf{pas le même centre} n'est \textbf{pas une rotation en général!} C'est le cas fréquemment, mais pas toujours.
\end{attention}

\begin{proof}
Considérons la rotation $f$ d'angle $\pi/2$ de centre l'origine, et la rotation $g$ d'angle $-\pi/2$ dont le centre est le point d'affixe $1$. Une figure suffit à se convaincre que la composée n'est pas une rotation mais une translation. Si ce n'est pas clair, on peut de toute façon le démontrer par le calcul : les rotations $f$ et $g$ sont représentées par 
\[
\tilde f : \begin{cases}\C\to\C\\z\mapsto e^{i\theta}(z-\omega)+\omega\end{cases}
\text{ et }
\tilde g : \begin{cases}\C\to\C\\z\mapsto e^{i\theta'}(z-\omega)+\omega\end{cases}
\]
La composée $f\circ g$ est représentée en coordonnée complexe par
\[
\tilde f\circ \tilde g : \begin{cases}\C \to \C\\ z\mapsto \tilde f (-iz+1+i) = i(-iz+1+i) = z-1+i\end{cases} 
\]
Ce n'est pas une rotation... en fait il s'agit dans ce cas d'une translation !
\end{proof}

Il existe néanmoins certains cas où la composée de deux rotations est une rotation, par exemple si elles ont le même centre, mais pas uniquement. Le résultat général est le suivant.

\begin{proposition}
Soient $f$ et $g$ deux rotations. Leur composée est:
\begin{enumerate}
\item Une translation si l'angle de $f$ est opposé à celui de $g$ modulo $2\pi$.
\item Une rotation dans tous les autres cas.
\end{enumerate}
\end{proposition}
\begin{proof}
On peut écrire $f$ et $g$ en coordonnées complexes : 
\[ \tilde f : \begin{cases}\C\to\C\\z\mapsto e^{i\theta}(z-\omega)+\omega\end{cases}
\text{ et }
\tilde g : \begin{cases}\C\to\C\\z\mapsto e^{i\theta'}(z-\omega')+\omega'\end{cases}
\]
On a alors:
\begin{align*}
\tilde f\circ \tilde g(z)
&= e^{i\theta}(\tilde g(z)-\omega)+\omega\\
&= e^{i\theta}((e^{i\theta'}(z-\omega')+\omega')-\omega)+\omega\\
&= e^{i(\theta+\theta')}(z-\omega')+e^{i\theta}(\omega'-\omega)+\omega.
\end{align*}
\begin{enumerate}
\item Si $e^{i\theta'}=e^{-i\theta}$, on voit la translation 
\[
\tilde f\circ \tilde g(z)=z+(\omega'-\omega)(e^{i\theta}-1)
\]
\item Sinon, on voit que la composée est une rotation d'angle $\theta+\theta'$. Le centre de cette rotation peut être déterminé, c'est laissé en exercice, mais on reviendra sur les points fixes des isométries et des similitudes directes dans la suite.
\end{enumerate}
\end{proof}

Enfin, la proposition suivante traite le cas des compositions \og mixtes \fg{}.

\begin{proposition}
La composée (dans un ordre quelconque) d'une translation et d'une rotation est toujours une translation ou une rotation.
\end{proposition}
\begin{proof}
Soit $f$ une translation et $g$ une rotation non triviales. Écrivons-les en coordonnées complexe : il existe $b\in\C$, $\omega\in\C$ et $\theta \in\R$ tels que
\[ \forall z\in\C, \tilde f(z)=z+b \text{ et } \forall z\in\C, \tilde g(z)=e^{i\theta}(z-\omega)+\omega\]
Écrivons maintenant ses composées : si $z\in\C$, on a 
\[ \tilde f\circ \tilde g(z) : (e^{i\theta}(z-\omega)+\omega)+b=e^{i\theta}z+\omega(1-e^{i\theta})\]
C'est une rotation d'angle $\theta$.

On procède de même pour l'autre composée (attention, le résultat n'est pas le même : c'est toujours une rotation d'angle $\theta$, mais potentiellement avec un centre différent).
\end{proof}




\chapter{Similitudes directes}
\minitoc
\hyperlink{toc}{\retourTOC}


\section{Définition, unicité de l'écriture complexe, cas particuliers}


\begin{definition}
Soit $f : \mathcal P\to \mathcal P$. On dit que c'est une similitude directe s'il existe $a\in\C^*$ et $b\in\C$ tels que
\[ \forall z\in\C, \tilde f(z)=az+b\]
\end{definition}


La définition précédente ne demande pas que les nombres complexes $a$ et $b$ soient uniques. En fait, s'ils existent, ils le sont automatiquement d'après le résultat suivant :

\begin{proposition}
Soient $a$, $a'$, $b$ et $b'$ des complexes tels que
\[ \forall z\in\C, az+b=a'z+b'\]
Alors $a=a'$ et $b=b'$.
\end{proposition}
\begin{proof}
En prenant $z=0$, on obtient $b=b'$. Comme $b=b'$ on obtient ensuite $\forall z\in\C, az=a'z$, et en prenant $z=1$ on obtient $a=a'$. 
\end{proof}

\begin{remarque}
D'après la proposition précédente, si $\phi$ est une similitude directe, les paramètres $a$ et $b$ de la définition sont forcément \textbf{uniques}.
\end{remarque}

\section{Cas particuliers : rotations, translations, homothéties}

\begin{proposition}
Les rotations, les translations sont des cas particuliers de similitudes directes.
\end{proposition}
\begin{proof}
D'après le cours sur les isométries classiques, une translation est de la forme $z\mapsto z+b$ donc est une similitude directe. De même, une rotation est de la forme $a\mapsto az+b$ (avec $|a|=1$, et $b=0$ si jamais $a=1$), c'est donc aussi une similitude directe.
\end{proof}


Il existe des similitudes qui ne sont pas des rotatins ou des translations. Par exemple, la transformation $z\mapsto 2z$ est manifestement une similitude directe, mais $0\mapsto 0$ et $1\mapsto 2$ ce qui montre que les distances ne sont pas conservées. Ce n'est donc pas une isométrie, et a fortiori ce n'est ni une rotation ni une rotation. Cette transformation appartient en fait à une autre grande famille de transformations classiques du plan : les homothéties, que l'on introduit ci-dessous.

\begin{definition}
Soit $f : \mathcal P \to \mathcal P$. On dit que $f$ est une \emph{homothétie} s'il existe $\Omega\in\mathcal P$ est $\lambda\in\R^*$ tels que
\[ \forall M\in\mathcal P, \overrightarrow{\Omega f(M)} = \lambda \overrightarrow{\Omega M}.\]

Autrement dit, $f$ est une homothétie s'il existe $\omega\in\C$ et $\lambda\in\R^*$ tels que
\[ \forall z\in\C, f(z)-\omega=\lambda(z-\omega).\]
\end{definition}

\begin{exemple}
La transformation du plan correspondant à $z\mapsto 2z$ est une homothétie. Celle correspondant à $z\mapsto -z/2+3i$ aussi. (Prendre $\lambda=-1/2$ et $\omega=2i$ : on vérifie en effet que $f(z)-\omega = f(z)-2i = -z/2+i=\frac{-1}{2}(z-2i)=\lambda(z-\omega)$.)
\end{exemple}

\begin{mdframed}
Attention, il existe des similitudes directes qui ne sont ni des translations, ni des rotations, ni des homothéties, par exemple l'application correspondant à $z\mapsto 2iz+1$.
\end{mdframed}

\section{Bijectivité, composition, réciproque}

\begin{proposition}[Les similitudes directes forment un groupe pour la composition]
\begin{enumerate}
\item La composée de deux similitudes directes est une similitude directe.
\item Une similitude directe est bijective et sa réciproque est une similitude directe.
\end{enumerate}
\end{proposition}
\begin{proof}
\begin{enumerate}
\item Soient $f$ et $g$ deux similitudes directes. Alors, il existe $a,a'\in\C^*$ et $b, b'\in\C$ tels que $f$ et $g$ soient représentées par $z\mapsto az+b$ et $z\mapsto a'z+b'$. L'application composée $f\circ g$ est donnée par 
\[ f\circ g : \C\to \C, z\mapsto f(g(z))=a(a'z+b')+b=(aa')z+(ab'+b)\]
\item Soient $a\in\C^*$, $b\in \C$. Si $z$ et $w$ sont des nombres complexes, on a
\[ w=az+b \iff z=\frac{w-b}{a}\]
Ceci signifie que l'application $z\mapsto az+b$ est bijective et que sa réciproque est l'application $w\mapsto a^{-1}w-ba^{-1}$, qui est une similitude directe.\\
Noter que l'on aurait alternativement pu exploiter la preuve du premier point, qui dit que si $f$ et $g$ sont des similitudes directes, alors :
\[
f\circ g = \Id 
\iff \left(aa'=1\text{ et } ab'+b=0\right) 
\iff \left(a'=a^{-1}\text{ et } b'=-ba^{-1}\right)
\]
\end{enumerate}
\end{proof}

Comme la composée de deux similitudes directes est une similitude directe, on en déduit en particulier que les composées de rotations, translations et homothéties sont toujours des similitudes directes.




%%%%%%%%%%%%%%%%%%%%%%%%%%%%
\section{Conservation des milieux, des barycentres, de l'alignement}

On a déjà vu qu'une similitude directe ne conserve en général pas les distances, puisqu'elle les multiplie par le rapport $k$. 

En revanche, elle conserve les barycentres et les angles orientés. Avant de parler de barycentres, on commence par un cas particulier dont la démonstration est simple

\begin{proposition}
Une similitude directe conserve les milieux.
\end{proposition}
\begin{proof}
Il s'agit de montre que pour toute similitude directe $f$ et tout segment $[PQ]$ de milieu $M$, alors $f(M)$ est le milieu du segment $\left[f(P)f(Q)\right]$. Pour le montrer on passe en coordonnée complexe : on note $\tilde f : z\mapsto az+b$  l'écriture en coordonnée complexe de $f$. L'affixe de $M$ est $m=\frac{p+q}{2}$, et celui du milieu $M'$ de $[f(P)f(Q)]$ est $m'=\frac{\tilde f(p)+\tilde f(q)}{2}$. On veut montrer que $M'=f(M)$ c'est-à-dire $m'=\tilde f(m)$. Or on a :
\[
m'
= \frac{\tilde f(p)+\tilde f(q)}{2}
=\frac{ap+b+aq+b}{2}
= a\frac{p+q}{2}+b 
=\tilde f(m)
\]
\end{proof}

La conservation des barycentres n'est qu'une version un peu plus évoluée du résultat précédent (le milieu de $P$ et $Q$ n'est autre que le barycentre de $P$et $Q$ avec coefficients $1/2$ et $1/2$). La preuve est semblable.

\begin{proposition}[Conservation des barycentres]
Soit $f$ une similitude directe. Alors, elle conserve les barycentres au sens suivant. Soit $(P_i)_{1\leq i\leq n}$ une famille de points du plan et $(\lambda_i)_{1\leq i\leq n}$ une famille de réels vérifiant $\sum_{i=1}^n\lambda_i=1$. Soit $P$ le parycentre des points $P_i$ avec coefficients $\lambda_i$, alors $f(P)$ est égal au barycentre $P'$ des points $f(P_i)$, affectés des mêmes coefficients $\lambda_i$.
\end{proposition}
\begin{proof}
La similitude directe $f$ est représentée par $\tilde f : z\mapsto az+b$ avec $a\in\C^*$ et $b\in\C$.
En coordonnée complexe, on a donc $p=\sum_{i=1}^n \lambda_ip_i$ et $p'=\sum_{i=1}^n \lambda_i\tilde f(p_i)$. On veut montrer que $p'=\tilde f(p)$. Pour cela on calcule $p'$ :

\[ 
p' = \sum_{i=1}^n\lambda_i\tilde f(p_i)
= \sum_{i=1}^n\lambda_i(ap_i+b)
= \sum_{i=1}^n a\lambda_ip_i + b\underbrace{\sum_{i=1}^n\lambda_i}_{=1}
= a\left(\sum_{i=1}^n \lambda_ip_i\right)+b
= ap+b 
= \tilde f(p)  
\]
\end{proof}

\begin{corollaire}
Soit $ABC$ un triangle, d'image $A'B'C'$ par une similitude directe $f$. Alors $f$ envoie le centre de gravité de $ABC$ sur le centre de gravité de $A'B'C'$.
\end{corollaire}
\begin{proof}
Le centre de gravité n'est autre que l'isobarycentre des trois sommets, autrement dit le barycentre des trois points avec coefficients $\frac13$.
\end{proof}

\begin{corollaire}
Une similitude directe conserve l'alignement, autrement dit elle envoie une droite sur une droite.
\end{corollaire}
\begin{proof}
Si $P$ et $Q$ sont deux points distincts, alors un point $M$ est aligné avec $P$ et $Q$ si et seulement c'est un barycentre de $P$ et $Q$.
\end{proof}


%%%%%%%%%%%%%%%%%%%%%%%%%%%%%%%%%%%%%%%%%
\section{Non-conservation des distances : rapport d'une similitude directe}

\begin{definition}
Soit $f$ une similitude directe, que l'on écrit $z\mapsto az+b$ en coordonnée complexe. Le rapport de $f$ est par définition le module de $a$. C'est un nombre réel strictement positif.
\end{definition}

\begin{exemple}
La similitude directe représentée par $z\mapsto -3z+2$ est de rapport $3$. Celle représentée par $z\mapsto (3+4i)z+3-i$ est de rapport $5$.
\end{exemple}

\begin{proposition}
Soit $f$ une similitude directe et $k\in\R_+^*$ son rapport. Alors $f$ \og multiplie toutes les distances par $k$\fg, ce qui signifie la chose suivante:
\[ \forall (M, M') \in\mathcal P^2, \dist(f(M),f(M'))=k\dist(M,M')\]
Ou encore, en coordonnée complexe:
\[ \forall (z, z')\in\C^2, \abs{\tilde f(z')- \tilde f(z')} = k\abs{z'-z}\] 
\end{proposition}
\begin{proof}
Les deux formulations sont équivalentes et on prouve donc la seconde. Soient $z$ et $z'$ deux nombres complexes. Alors on a 
\begin{align*}
\abs{\tilde f(z')- \tilde f(z')}
&= \abs{az'+b-(az+b)}\\
&= \abs{a(z'-z)}\\
&=\abs{a}\cdot\abs{z'-z}\\
&= k\abs{z'-z}
\end{align*}
\end{proof}

\begin{remarque}
Soit $f$ une similitude directe de rapport $k\in\R_+^*$. C'est une isométrie si et seulement si $k=1$.
\end{remarque}

\begin{comment}
\begin{attention}
La terminologie \emph{rapport} désigne deux choses différentes selon si on parle d'une homothétie ou bien d'une similitude. C'est un peu malheureux mais c'est comme ça.
\begin{enumerate}
\item Le rapport d'une homothétie est un réel non nul. Par exemple, une homothétie peut avoir rapport $-1$.
\item Le rapport d'une similitude est un réel strictement positif. Une similitude ne peut pas avoir un rapport $-1$.
\end{enumerate}
Le problème est qu'une homothétie est un cas particulier de similitude. Considérons par exemple une symétrie centrale $f$, autrement dit une rotation d'angle $\pi$, ou encore, une \textbf{homothétie de rapport $-1$}. L'application $f$ est également une similitude, et \textbf{en tant que similitude}, son rapport est... $1$ (et pas $-1$).

Avec l'habitude, ceci ne pose pas de problème mais si on n'est pas assez familier des différentes notions, on peut confondre les contextes et faire des fautes. Conclusion : faites des exercices d'entraînement !
\end{attention}

\begin{remarques}
Considérons une similitude directe $f$. D'après ce qui précède, on peut attacher à $f$ deux paramètres uniques $a\in\C^*$ et $b\in\C$ qui sont ceux de la définition de similitude directe. La situation est alors résumée par le tableau suivant:\\

\begin{tabular}{|l|c|c|c|}\hline
			& $a=1$	& $|a|=1$ et $a\neq 1$	& $|a|\neq 1$	\\ \hline
$b=0$		& Identité (translation et rotation) & rotation non triviale &  \\ \hline
$b\neq 0$	& translation non triviale & rotation non triviale & \\ \hline 
\end{tabular}
\end{remarques}

\end{comment}


%\begin{proposition}
%Soit $f$ une similitude directe, que l'on écrit $z\mapsto az+b$ en coordonnée complexe. Alors $f$ est une isométrie si et seulement si $|a|=1$.
%\end{proposition}
%\begin{proof}
%Si $|a|=1$, on a vu que $f$ est  une isométrie.
%
%Réciproquement si $|a|\neq 1$, considérons les points $M$ et $M'$ d'affixes $z=0$ et $z'=1$ : ils sont à distance $1$. Par contre, les points $f(M)$ et $f(M'$ ont pour affixes $az+b=b$ et $az'+b = a+b$. Ils sont à distance $|f(z')-f(z)| = \abs{a+b-b}=|a|\neq 1$.

%On en déduit que $f$ ne conserve pas les distances, et donc n'est pas une isométrie.
%\end{proof}


\section{Points fixes des similitudes directes}

\begin{proposition}
Soit $f :\mathcal P\to \mathcal P$ une similitude directe, et soient $a\in\C^*$ et $b\in\C$ tels que 
\[ \forall z\in\C, \tilde f (z) = az+b\]
\begin{enumerate}
\item Si $a\neq 1$, alors $f$ admet un unique point fixe $\Omega$, d'affixe $\omega=\frac{b}{1-a}$.
\item Si $a=1$, alors $f$ est une translation. Dans ce cas, il y a encore deux sous-cas possibles:
\begin{enumerate}
\item Si $a=1$ et $b=0$, l'application $f$ est l'identité et tous les points sont fixes.
\item Si $a=1$ et $b\neq 0$, l'application $f$ est une translation non triviale, et aucun point n'est fixe.
\end{enumerate}
\end{enumerate}
\end{proposition}
\begin{proof}
\begin{enumerate}
\item Si $a\neq 1$, alors considérons un point $M$ d'affixe $z$. Alors on a la chaîne d'équivalences:
\begin{align*}
M\text{ est fixe}
&\iff f(M)=M& \text{(définition de point fixe)}\\
&\iff \tilde f(z)=z & \text{(passage en coordonnée complexe)}\\
&\iff az+b=z & \text{(écriture concrète)} \\
&\iff z=\frac{b}{1-a}\quad \text{ car }a\neq 1
\end{align*}
\item Si $a=1$, alors $b$ peut être nul ou bien non nul :
\item 
\begin{enumerate}
\item Si $a=1$ et $b=0$, l'application $\tilde f$ est $\tilde f : \C\to\C, z\mapsto z$ autrement dit c'est l'identité de $\C$, et donc $f$ est l'identité de $\mathcal P$. Tous les points sont fixes.
\item Si $a=1$ et $b\neq 0$, l'application $\tilde f$ est $\tilde f : \C\to\C, z\mapsto z+b$. Elle n'a aucun point fixe puisque l'équation $z=z+b$ n'a aucune solution.
\end{enumerate}
\end{enumerate}
\end{proof}


\begin{definition}
Soit $f$ une similitude directe. On dit que $f$ a un centre, ou encore, que $f$ est une similitude à centre, si elle a un unique point fixe, autrement dit si $a\neq 1$. Dans ce cas, son centre est par définition son unique point fixe.
\end{definition}

\section{Conservation des angles orientés}


\begin{proposition}[Conservation des angles orientés]
Soit $f$ une similitude directe, et $P$, $Q$ et $R$ trois point distincts. Alors
\[ \widehat{(\overrightarrow{PQ}, \overrightarrow{PR})}
\equiv \widehat{(\overrightarrow{f(P)f(Q)}, \overrightarrow{f(P)f(R)})} \mod 2\pi\]
Autrement dit, en coordonnée complexe, si $p$, $q$ et $r$ sont trois complexes distincts, alors:
\[ \arg\frac{r-p}{q-p} \equiv \arg \frac{\tilde f(r)-\tilde f(p)}{\tilde f(q)-\tilde f(p)} \mod 2\pi\]
\end{proposition}
\begin{proof}
Les deux formulations sont équivalentes donc on ne démontre que la seconde. Écrivons $\tilde f : z\mapsto az+b$ la similitude en coordonnée complexe. Alors
\[ \frac{\tilde f(r)-\tilde f(p)}{\tilde f(q)-\tilde f(p)}
=\frac{ar+b-(ap+b)}{aq+b-(ap+b)}
=\frac{a(r-p)}{a(q-p)}
=\frac{r-p}{q-p}
\]
Donc les deux nombres complexes sont égaux et non nuls. Ils ont donc le même argument modulo $2\pi$.
\end{proof}

\begin{remarque}
La proposition précédente donne une nouvelle démonstration de la conservation de l'alignement, puisque $P$, $Q$ et $R$ sont alignés ssi $\widehat{(\overrightarrow{PQ}, \overrightarrow{PR})}\equiv 0\text{ ou } \pi \mod 2\pi$.
\end{remarque}


Voici quelques exemples pour mieux comprendre la propriété de conservation des angles et la préservation des rapports de distances:


\begin{exemples}
\begin{enumerate}
\item Une similitude directe envoie un triangle équilatéral direct sur un triangle équilatéral direct (mais éventuellement de taille et orientation différentes).
\item Une similitude directe envoie un carré direct sur un carré direct (mais éventuellement de taille et orientation différentes). Elle envoie deux droites perpendiculaires sur deux autres droites perpendiculaires.
\item Une similitude envoie un repère ortho\textbf{normé direct} sur un repère ortho\textbf{gonal direct} (mais pas normé : l'angle droit direct est conservé, mais pas les distances, qui sont toutes multipliées par le rapport $k$).
\item Une similitude directe envoie un triangle sur un autre triangle ayant les mêmes angles, dans le même ordre (mais éventuellement de taille et orientation différentes). D'ailleurs, deux tels triangles sont dits \textbf{directement semblables}.
\end{enumerate}
\end{exemples}
En ce qui concerne les angles, on a une propriété un peu plus forte que simplement la conservation des angles : l'existence d'un angle absolu qui est un élément caractéristique de la similitude. Ceci est expliqué dans le paragraphe suivant.


\section{Angle d'une similitude directe}

\begin{definition}
Soit $f :\mathcal P\to \mathcal P$ une similitude directe, et soient $a\in\C^*$ et $b\in\C$ tels que 
\[ \forall z\in\C, \tilde f (z) = az+b\]
Soit $\theta\in\R$ un argument de $a$. Sa classe de congruence modulo $2\pi$ est \emph{l'angle} de la similitude $f$.
\end{definition}

\begin{exemple}
La similitude correspondant à $z\mapsto (2+2i)z+4-3i$ a un angle égal à  $\pi/4$ (modulo $2\pi$). 
\end{exemple}

La terminologie est justifiée par le résultat suivant :

\begin{proposition}
Soit $f$ une similitude directe d'angle $\theta$ (modulo $2\pi$).
\[ \forall (M,P)\in\mathcal P^2, M\neq P\implies \widehat{(\overrightarrow{MP}, \overrightarrow{f(M)f(P)})} \equiv \theta \mod 2\pi.\]
De façon équivalente:
\[ \forall (m,p)\in\C^2, m\neq p \implies \arg \frac{\tilde f(p)-\tilde f(m)}{p-m}\equiv \theta \mod 2\pi\]
\end{proposition}
\begin{proof}
Les deux formulations sont équivalentes donc on ne démontre que la seconde. Écrivons $\tilde f : z\mapsto az+b$ la similitude en coordonnée complexe. Soient $m$ et $p$ des complexes distincts. Alors, on a 
\[
\frac{\tilde f(p)-\tilde f(m)}{p-m}
= \frac{ap+b-(am+b)}{p-m}
= \frac{a(p-m)}{p-m}
= a
\]
\end{proof}

En d'autres termes, il y a toujours le même angle $\theta$ entre deux points quelconques et leurs deux images : l'angle est une grandeur caractéristique de la similitude.

\section{Écriture d'une similitude à centre comme composée de rotation et d'homothétie de même centre}

Soit $f$ une similitude directe, s'écrivant $z\mapsto az+b$. Écrivons de plus $a$ sous forme exponentielle : $a=re^{i\theta}$. On peut évidemment écrire cette application comme la composée suivante:
\[ z \mapsto e^{i\theta} z\mapsto re^{i\theta}z = az\mapsto az+b\]
Autrement dit, on écrit $z\mapsto az+b$ comme la composée successive de trois applications : d'abord $z\mapsto e^{i\theta}z$ autrement dit la rotation de centre $0$ et d'angle $\theta$, suivie d'une homothetie $z\mapsto rz$ (toujours centrée sur l'origine), suivie d'une translation $z\mapsto z+b$.

Cependant, cette décomposition n'est pas très satisfaisante, pour plusieurs raisons, en particulier parce qu'elle fait jouer un rôle spécial à l'origine alors que ce point n'a sans doute rien de particulier vis-à-vis de $f$.

Il arrive qu'un point ait effectivement un rôle particulier : lorsque ce point est l'unique point fixe de $f$. On a aalors le résulatt de décomposition suivant :

\begin{proposition}
Soit $f$ une similitude directe à centre, c'est-à-dire une similitude directe admettant un unique point fixe que l'on note $\Omega$. Alors $f$ peut s'écrire comme la composée d'une rotation de centre $\Omega$ et d'une homothétie de centre $\Omega$.
\end{proposition}



%%%%%%%%%%%%%%%%%%%%%%%%%%%%
\section{Action sur les points et sur les couples de points}

\begin{proposition}
Soient $M$ et $P$ deux points du plan. Alors il existe une similitude $f$ telle que $f(M)=P$. 
\end{proposition}
\begin{proof}
On a déjà montré précédemment qu'il existe une infinité d'isométries directes ayant cet effet. Comme une isométrie est un cas particulier de similitude directes, ceci prouve le résultat.
\end{proof}

\begin{proposition}
Soient $P$, $Q$, $R$ et $S$ quatre points du plan, d'affixes $p$, $q$, $r$ et $s$. Si $p\neq r$ et $q\neq s$, alors il existe une unique similitude directe $f : \mathcal P\to \mathcal P$ vérifiant
\[ f(P)=Q \text{ et } f(R)=S\]
\end{proposition}
\begin{proof}
Soit $f$ une similitude directe, que l'on écrit $z\mapsto az+b$ en coordonnée complexe. Les deux conditions $f(P)=Q \text{ et } f(R)=S$ correspondent au système d'équations
\[ \left\{\begin{matrix}
ap&+&b&=&q\\
ar&+&b&=&s
\end{matrix}\right.\]
Dans ce système, les inconnues sont $a$ et $b$, et $p$, $q$, $r$ et $s$ sont des paramètres fixés par l'énoncé. On le résout par la méthode classique, par opérations sur les lignes. 
\begin{enumerate}
\item L'opération $L_2-L_1$ donne l'équation $a(r-p)=s-q$. Comme $r\neq p$, on en tire $a=\frac{s-q}{r-p}$.
\item D'autre part l'opération $rL_1-pL_2$ donne l'équation $rb-pb=rq-ps$ et toujours comme $r-p\neq 0$, on obtient $b=\frac{rq-ps}{r-p}$.
\end{enumerate}
On obtient donc une solution unique $(a,b)$. Il reste à remarquer que comme $s\neq q$, la solution obtenue vérifie $a\neq 0$. Les paramètres $a$ et $b$ correspondent donc bien à une similitude directe.
\end{proof}

\begin{remarque}
Il est contre-productif d'essayer de retenir par c\oe ur les deux formules qui donnent $a$ et $b$ en fonction des quatre autres paramètres. La probabilité de confondre les différents paramètres est trop haute. Il vaut mieux refaire le calcul à chaque fois.
\end{remarque}

\begin{exercice}
Soient $A$, $B$, $C$ et $D$ les points d'affixes $a=1+i$, $b=2-i$, $c=3i$ et $d=-2+5i$.
Déterminer la similitude directe qui envoie $A$ sur $D$ et $B$ sur $C$. (Attention à l'ordre et au nom des paramètres : écrire la similitude $z\mapsto \alpha z + \beta$ par exemple.)
\end{exercice}



\appendix
\chapter{Annexes}

\newpage
\section{Alphabet grec}

\noindent Les lettres de l'alphabet grec sont couramment employées en mathématiques. Il est donc indispensable de bien les connaître.

\begin{center}

\def\arraystretch{1.2}
\setlength\tabcolsep{20pt}
\begin{tabular}{llll}
Alpha		& A 			& $\alpha$\\
Bêta			& B 			& $\beta$\\
Gamma		& $\Gamma$ 	& $\gamma$\\
Delta		& $\Delta$ 	& $\delta$\\
Epsilon		& E 			& $\varepsilon$\\
Zêta			& Z 			& $\zeta$\\
Êta			& H 			& $\eta$\\
Thêta		& $\Theta$ 	& $\theta$\\
Iota			& I 			& $\iota$\\
Kappa		& K 			& $\kappa$\\
Lambda		& $\Lambda$ 	& $\lambda$\\
Mu			& M 			& $\mu$\\
Nu			& N 			& $\nu$\\
Ksi			& $\Xi$ 		& $\xi$\\
Omicron		& O 			& o\\
Pi			& $\Pi$ 		& $\pi$\\
Rhô			& P 			& $\rho$\\
Sigma		& $\Sigma$ 	& $\sigma$\\
Tau			& T 			& $\tau$\\
Upsilon		& $\Upsilon$& $\upsilon$\\
Phi			& $\Phi$ 	& $\phi$, $\varphi$\\
Khi			& X 			& $\chi$\\
Psi			& $\Psi$ 	& $\psi$ \\
Oméga		& $\Omega$ 	& $\omega$\\
\end{tabular}
\end{center}


\end{document}